%!TEX encoding = UTF-8 Unicode
% --------------------------------------------------------------------------------
\documentclass[
	fontsize=12pt,
	headings=small,
	parskip=half,           % Ersetzt manuelles Setzen von parskip/parindent.
	bibliography=totoc,
	numbers=noenddot,       % Entfernt den letzten Punkt der Kapitelnummern.
	open=any,               % Kapitel kann auf jeder Seite beginnen.
%	final                   % Entfernt alle todonotes und den Entwurfstempel.
]{scrreprt}
% --------------------------------------------------------------------------------
% Hinweis: Das Übersetzen dieser Datei funktioniert auch Online mit 
% https://www.overleaf.com. Hierzu müssen neben dieser Datei im gleichen 
% Verzeichnis die Dateien hinweiseabschlussarbeit.bib und stylesvs.tex liegen.
% --------------------------------------------------------------------------------
\input{stylesvs}

\title{Wirtschaftsinformatik und Nachhaltigkeit:\\
Anwendungsszenarien in der Transportwirtschaft}
\author{Valentina Ermisch, Lisa-Sophie Kaisik}
% \date{01.01.2015} % Für bestimmtes Datum diese Zeile aktivieren.

\begin{document}

\newpage
\thispagestyle{empty}
\null
\newpage

% --------------------------------------------------------------------------------
% ----- Muster für weitere Deckblätter siehe hinten
% --------------------------------------------------------------------------------
\begin{titlepage}
\mbox{\parbox[t][1.75cm][b]{2.2cm}{\uhhlogo}}
\begin{center}\Large
	\vfill Seminararbeit
	\vfill \makeatletter {\Large\textsf{\textbf{\@title}}\par} \makeatother
	\vfill vorgelegt von \par\bigskip
	\begin{tabu}{p{0.5\textwidth}p{0.5\textwidth}}
	\centering Valentina Ermisch     & \centering Lisa-Sophie Kaisik \\[1ex]
	\centering Matrikelnummer 1234567 & \centering Matrikelnummer 7726396 \\
	\centering Studiengang Betriebs- & \centering Studiengang Wirtschafts- \\
	\centering wirtschaftslehre & \centering informatik \\

	\end{tabu}
	\vfill MIN-Fakultät \par Fachbereich Wirtschaftsinformatik
	\vfill \makeatletter eingereicht am {\@date} \makeatother
	\vfill Betreuer: Prof. Dr. Markus Nüttgens
\end{center}
\ifoptionfinal{}{
\begin{tikzpicture}[remember picture, overlay]
	\node[draw, red, font=\ttfamily\bfseries\Large, xshift=30mm, yshift=238mm, rotate=340, text centered, text width=6cm, very thick, rounded corners=4mm] at (current page.south) {Entwurf vom \today};
\end{tikzpicture}
%\begin{tikzpicture}[overlay]
%	\node[draw, blue, font=\sffamily\Large, xshift=0mm, yshift=242mm, rotate=0, text centered, rounded corners=1mm] at (current page.south) {Muster des Deckblatts für Seminararbeiten};
%\end{tikzpicture}
}
\end{titlepage}

% --------------------------------------------------------------------------------
% ----- Ab hier folgt der Haupttext
% --------------------------------------------------------------------------------

%\chapter*{Aufgabenstellung}

%Soweit eine ausformulierte Aufgabenstellung mit der Betreuerin bzw. dem Betreuer vereinbart wurde, diese bitte hier einfügen.

%\chapter*{Vorwort, Zusammenfassung}

%Für die eilige Leserin bzw. den eiligen Leser sollen auf etwa einer halben, maximal einer Seite die wichtigsten Inhalte, Erkenntnisse, Neuerungen bzw. Ergebnisse der Arbeit beschrieben werden.

%Durch eine solche Zusammenfassung (im Engl. auch Abstract genannt) am Anfang der Arbeit wird die Arbeit deutlich aufgewertet. Hier sollte vermittelt werden, warum man die Arbeit lesen sollte.

\pagenumbering{roman}

\tableofcontents

\cleardoublepage
\pagenumbering{arabic}
\setcounter{page}{1}

\chapter{Einleitung}

\section{Motivation und Problemstellung}

Von Jahr zu Jahr wird es immer wärmer, weshalb es wichtiger denn je wird, den Zielen den Pariser Klimaabkommen näherzukommen. Das Ziel ist es, den Temperaturanstieg unter 2°C zu halten. Dazu ist es besonders wichtig, die CO2-Emissionen zu senken, die zu einem Großteil durch den Transportsektor verursacht werden. 
Neben den klimatischen Herausforderungen entwickelt sich unsere Gesellschaft immer weiter. Die Kundenanforderungen steigen und die Komplexität der Logistik wächst. Der Transportsektor steht daher vor der Herausforderung, nachhaltige Lösungen entlang logistischer Prozesse zu entwickeln. Dabei soll nicht nur effizient, sondern auch nachhaltig weiterentwickelt werden. 
Ein zentraler Aspekt zur Gestaltung effizienter und nachhaltiger Transportwirtschaft können verschiedene Lösungsansätze wie optimierte Routen durch KI, maschinelles Lernen oder intelligente Transportsysteme sein. Das schafft ökologische sowie ökonomische Vorteile.

\section{Zielsetzung}

Das Ziel dieser Seminararbeit ist zu erklären, wie IT-gestützte Logistik mehr Potential zur Realisierung von Effizienz, Kosteneffektivität und Nachhaltigkeit beiträgt, um den steigenden Anforderungen gerecht zu werden. Es sollen mögliche Anwendung von Transportmanagementsystemen, intelligenten Transportsystemen und Lösungen durch künstliche Intelligenz dargestellt werden. Dabei soll ihr Nutzen für die Weiterentwicklung der Transportwirtschaft deutlich werden.

\section{Aufbau der Semianrarbeit}

Zunächst werden die Grundlagen der Bereiche Transportwirtschaft, Nachhaltigkeit und IT-Technologien der Wirtschaftsinformatik erläutert. Es werden Herausforderungen der Transportwirtschaft aufgeführt, die aus unserer stetig wachsenden Gesellschaft resultieren. Der Bereich Nachhaltigkeit zeigt die Anforderungen an zukünftige Logistik, führt Ziele auf und unterscheidet zwischen Grüner Logistik und nachhaltiger Logistik. Die Grundlagen der Wirtschaftsinformatik definieren erste Begrifflichkeiten für ein Grundverständnis des Hauptteils. Es geht um die Begriffe, E-Logistik, Informations- und Kommunikationstechnologien (IKT), Smart Logistics, Logistik 4.0, automatisierte und intelligente Verwaltung, sowie Technologien wie Internet of Things (IoT), Big Data, Cloud Computing, künstliche Intelligenz (KI) und Blockchain.

Im darauf folgenden Hauptteil werden die einzelnen Möglichkeiten zur Steigerung von Effizienz und Nachhaltigkeit in der Transportwirtschaft genauer aufgeführt. Es geht um strukturierte Routenoptimierung durch Transportation Management Systeme (TMS). Es wird der Einsatz von intelligenten Transportsystemen beschrieben, welche ihre Daten aus einem System namens Internet of Things (IoT) ziehen. Zuletzt geht es um den Einsatz von Blockchain-Technologien und Smart Contracts in Lieferketten. 

Für die konkrete Anwendung der aufgezeigten Technologien wird die Umsetzung der Lösungen durch das Unternehmen Amazon Web Service Inc. erläutert und bewertet. Es wird nach den drei Säulen der Nachhaltigkeit und einer Abwägung zwischen Problemlösungen und neu entstehenden Problemen abgewogen.

Zuletzt folgt eine Zusammenfassung der Kernergebnisse und eine Auswertung, ob die genannten IT Lösungen einen Beitrag zur effizienten und nachhaltigen Entwicklung des Transportsektors beitragen. Im Ausblick werden zukünftige Fragen und neue Herausforderungen, die sich durch die Nutzung von IT-Lösungen ergeben, aufgeführt.


\chapter{Grundlagen}


\section{Transportwirtschaft und ihre Herausforderungen}

\todo{Definition Logistik: System zur optimalen Versorgung mit Materialien}
\todo{duale Strukturiertheit der Logistik, raum-Zeit-Güter und die Bedeutung des Informationsmanagements }
\todo{8 Rs der Logsitik, grundlegendes Ziel}
\todo{Einordnung und Abgrenzung zum Supply Chain Managemenr (SCM), wobei die Logistik Teildisziplin ist}

\section{Nachhaltigkeit mit Bezug auf Transportwirtschaft}


Der Begriff der Nachhaltigkeit besteht aus drei Hauptkomponenten: aus der ökologisch, der ökonomisch und der sozialen. Die ökologische Nachhaltigkeit beinhaltet den Schutz des Klima, der Biodiversität und die bedachte Nutzung der Ressourcen. Die ökonomische Komponente zeichnet den bedachten Konsum von Gütern aus. Die soziale Nachhaltigkeit setzt auf  Bildung, Gesundheit und Chancengleichheit für jeden. \cite{Greenpeace2025}

Agenda 2030 setzt sich mit der Nachhaltigkeit auseinander. Dabei umfasst diese 17 Ziele, auch Sustainable Development Goals oder SDGs genannt. Diese traten am 01.01.2016 in Kraft und erweiterten somit die Ziele der vorherigen Agenda 21. 193 UN-Mitgliedstaaten haben diese verabschiedet, jedoch ist die Agenda 2030 nur eine (freiwillige) Selbstverpflichtung für die jeweiligen Länder. \cite{LpB-BW2023}

SDG 7, 12 und 13 sind die drei Ziele, die einen direkten Bezug zum Klimaschutz haben. Dabei schreibt das SDG 7 “Bezahlbare und saubere Energie”, dass bis 2030 jeder Zugang zu nachhaltiger und moderner Energie haben sollte. Darüber hinaus muss die Energie bezahlbarer und verlässlicher werden. \cite{LpB-BW2023}
Das SDG 12 “Nachhaltige/r Konsum und Produktion” setzt voraus, dass der Mensch einem verantwortungsvollen Konsum nachgeht und nicht im Überfluss lebt. Dabei wird auf eine funktionale Kreislaufwirtschaft gesetzt. (17 Ziele)
Der Klimaschutz wird mit dem SDG 13 "Maßnahmen zum Klimaschutz” abgedeckt. Dieses schreibt vor, dass bis 2030 die Treibhausgasemissionen um 42 Prozent gesenkt werden müssen, damit die globale Erwärmung nicht über 1,5 Grad Celsius steigt. (17 Ziele)


\todo{Tripple Bottom Line, drei Säulen}
\todo{Definition und Untrscheidung Green Logistic (Fokus auf nur Ökologie, CO2) und Nachhaltiger Logistik (alle drei Säulen)}
\todo{Notwendigkeit der Quantifizierung der Auswirkunge, ISO14083 Berechnung von Treibhausgas Emissionen THG in Transportketten}
\todo{Einsatz von IKT im Kontext von Reduzierung der Treibhausgase}

\newpage
\section{Wirtschaftsinformatik und Smart Logistics}

\todo{Definition (E-Logsitik): Planung, Steuerung Überwachung der Flüsse als Business Lösung / System}
\todo{IT IKT erfolgreichste Möglichkeit mit wieterbringenden Lösungen}
\todo{Definition Smart Logistics für Logistik 4.0}
\todo{Optimierung, Erreichung von Nachhaltigekti durch automatisierte und intelligente Verwaltung logischer Operationen}
\todo{Alle möglichen technologischen Treiber einführen: IoT, Big Data, Cloud Computing, KI}

\chapter{Anwendung nachhaltiger IT-gestützten Lösungsansätze}

\section{Effizienzsteigerung durch Management Systeme (TMS) und Routenoptimierung}

\subsection{Funktionsweise und Mehrwert}

\todo{Routenoptimierung und Tourenplanung, Bündlung von Einzelfahrten, Reduktion der Fahrtenanzahl und damit Reduktion von CO2 Ausstoß. Komplexität gesteruert durch operations Research}

\subsection{Rolle von KI und Big Data in der Routenoptimierung}

\todo{BIg Data zur Auswertung von Massendaten und Entscheidungsunterstützung. Optimierung von Kapazitätsauslastungen durch Big Data Analysen. KI als Lösung für Treibhausneutralität}

\newpage

\section{Intelligente Transportsysteme (ITS) und Internet of Things (IoT)}

\todo{Verschmelzung von material und Informationsfluss, Schaffung von interconnected and intelligent networks, Echtzeit Tracking und automatisierter Entscheidungsfindung}
\todo{Verbesserung des NAchhaltigkeit von Transportsystemen}

\newpage

\section{Blockchain Technologien in Lieferketten}

\todo{schnelle und transparante Abwicklung in weltweiten Lieferketten}
\todo{Effizient, Kapazität und Zuverlässigkeit im Transport und SCM}
\todo{Smart Contracts - automatisierte Initiierung von Aktivitäten}



\chapter{Anwendungsbeispiel}

\todo{Ausreichende Quellen für ein Unternehmen finden, dass wir als Beispiel nehmen wollen}
\todo{Daimler aktuell zu wenig Quellen, AWS aktuell zu viel und verschleiert / Seriösität?}


\section{Amazon Web Services Inc.}



\section{Bewertung}


\chapter{Fazit}

\todo{Zusammenfassung Kernergebnisse, Beantwortung Fragestellung}
\todo{IT gestützte Logistik Sicherung einer nachhaltigen Unternehmensentwicklung}
\todo{Zusammenfassend Bestätigen, dass Logistik und IT Implementierung zusmamengehört}


\chapter{Ausblick}

\todo{ungelöste Probleme, zukünftige Fragen, weitere quantifizierung der IT Lösungen}
\todo{IT Sicherheit! Cyber Security, Risiken}
\todo{Green IT: Bringt das wirklich Energieeinsparungen oder werden diese nut zur IT weitergetragen}


% --------------------------------------------------------------------------------
% ----- Literaturverzeichnis
% --------------------------------------------------------------------------------
\begin{raggedright} % raggedright schaltet den Blocksatz ab und erzeugt ein stimmigeres Schriftbild im Literaturverzeichnis.
  \printbibliography % alphabetic ist definiert unter biblatex in style.svs
  \label{sec:literaturverzeichnis}
\end{raggedright}

% --------------------------------------------------------------------------------
% ----- Anhang
% --------------------------------------------------------------------------------
\appendix
\setcounter{figure}{0}
\renewcommand\thetable{A.\arabic{figure}}
\setcounter{table}{0}
\renewcommand\thetable{A.\arabic{table}}

% --------------------------------------------------------------------------------
% ----- Eidesstattliche Versicherung
% --------------------------------------------------------------------------------
\chapter*{Eidesstattliche Versicherung}
\vspace{1cm}

Ich erkläre eidesstattlich, dass ich die Arbeit selbständig angefertigt, keine anderen als die angegebenen Hilfsmittel benutzt und alle aus ungedruckten Quellen, gedruckter Literatur oder aus dem Internet im Wortlaut oder im wesentlichen Inhalt übernommenen Formulierungen und Konzepte gemäß den Richtlinien wissenschaftlicher Arbeiten zitiert, durch Fußnoten gekennzeichnet bzw. mit genauer Quellenangabe kenntlich gemacht habe.

Ich versichere, dass auch im Anwendungsfall von generativer Künstlicher Intelligenz (genKI) meine eigene schöpferische Leistung der erhebliche Anteil in dieser Seminararbeit ist und ich die genutzte genKI detailliert in einem Anhang in meiner Seminararbeit aufgeführt und die Zitate in der Seminararbeit deutlich gekennzeichnet habe. Dieser Anhang ist Teil meiner Seminararbeit. Ich bin für ggfs. durch genKI generierte Inhalte, die Einhaltung urheberrechtlicher Bestimmungen, meine eigenständige Erstellung sowie für die wissenschaftliche Integrität meiner Seminararbeit selbst verantwortlich. Mir ist bekannt, dass fehlende oder fehlerhafte Angaben als Täuschungsversuch gewertet werden können. Ich erkläre, dass ich die Bestimmungen zum Urheberrecht und Datenschutz (DSGVO) sowie die jeweils geltenden Richtlinie der Fakultät für Wirtschaftsinformatik zur Anwendung von genKI-Tools erfüllt habe und erfüllen werde.

\makeatletter
Hamburg, den {\@date}
\makeatother

\vspace{2cm}
\rule{6cm}{0.25pt}\\
\makeatletter
Valentina Ermisch \par
\makeatother

\vspace{2cm}
\rule{6cm}{0.25pt}\\
\makeatletter
Lisa-Sophie Kaisik \par
\makeatother

\newpage
\thispagestyle{empty}
\null
\newpage

% --------------------------------------------------------------------------------
% ----- Literaturliste (Muster)
% --------------------------------------------------------------------------------
\newpage
\thispagestyle{empty}
\label{sec:literaturliste}
\par\textbf{\textsf{Thema:}} Wirtschaftsinformatik und Nachhaltigkeit: Anwendungsszenarien in der Transportwirtschaft
\par\textbf{\textsf{Bearbeiter:}} Valentina Ermisch, Lisa-Sophie Kaisik
\par\textbf{\textsf{Datum:}} \today
\bigskip
% % ====> Delete me
% \begin{tikzpicture}[overlay]
%     \node[draw, blue, font=\sffamily\Large, xshift=80mm, yshift=-6mm, rounded corners=1mm]{Muster der Literaturliste};
% \end{tikzpicture}
% % <==== /Delete me

\section*{Literaturliste}

% ----- Nachfolgend eine händisch gesetzte Literaturliste, die sich exakt an die Syntax im Abschnitt \ref{sec:literaturhowto} hält. Wir nutzen diese aber hier nicht, sondern lassen BibLaTeX die Einträge formatieren.
\iffalse
David Chaum: Untraceable Electronic Mail, Return Addresses, and Digital Pseudonyms. Communications of the ACM 24/2 (1981) 84--88.

David Chaum: The Dining Cryptographers Problem: Unconditional Sender and Recipient Untraceability. Journal of Cryptology 1/1 (1988) 65--75.

David Goldschlag, Michael Reed, Paul Syverson: Onion Routing for Anonymous and Private Internet Connections. Communications of the ACM 42/2 (1999) 39--41.

Andreas Pfitzmann: Diensteintegrierende Kommunikationsnetze mit teilnehmerüberprüfbarem Datenschutz. IFB 234, Springer-Verlag, Berlin 1990.

Wei Wang, Mehul Motani, Vikram Srinivasan: Dependent link padding algorithms for low latency anonymity systems. Proc. 15th ACM conference on Computer and communications security. ACM, 2008, 323--332.
\fi

% ----- Nachfolgend die Ausgabe unter Verwendung von BibLaTeX. Die Formatierung übernimmt BibLaTeX. Dadurch wird es zu Abweichungen von der vorgegebenen Syntax kommen. Dies ist tolerabel, da es i.W. auf Einheitlichkeit ankommt, nicht auf eine dogmatische Einhaltung der Syntax.
\fullcite{AnalysisTransportSDGs}

\fullcite{RIOHGIZ2022}

\fullcite{Hausladen2020}

\fullcite{HLAG2016}

\fullcite{Fareed2024}

\fullcite{UNCTAD2018}

\fullcite{UN2021}

\fullcite{UBA2024}

\fullcite{WEF2025}

\fullcite{BMZ2025}

\fullcite{Greenpeace2025}

\fullcite{UNRIC2025}

\fullcite{Bundestag2025}

\fullcite{AmmoserHoppe2006}

\fullcite{LpB-BW2023}

\fullcite{UBA2025}


% --------------------------------------------------------------------------------
% ----- Wiss. Kurzzusammenfassung (Muster)
% --------------------------------------------------------------------------------
%\newpage
%\thispagestyle{empty}
%\label{sec:kurzusammenfassung}
%\par\textbf{\textsf{Thema:}} Privacy Enhancing Technologies zum Schutz von Kommunikationsbeziehungen
%\par\textbf{\textsf{Bearbeiter:}} Eva Musterfrau, Heinz Mustermann
%\par\textbf{\textsf{Datum:}} \today
%\bigskip
% ====> Delete me
%\begin{tikzpicture}[overlay]
	%\node[draw, blue, font=\sffamily\Large, xshift=60mm, yshift=-6mm, rounded corners=1mm]{Muster der Wiss. Kurzzusammenfassung};
	%\node[font=\sffamily\small\itshape, xshift=72mm, yshift=-14mm]{Umfang: 1-3 Seiten, wenn keine konkrete Vorgabe};
%\end{tikzpicture}
% <==== /Delete me
%\section*{Überschrift}

%Lorem ipsum dolor sit amet, consectetur adipisicing elit, sed do eiusmod tempor incididunt ut labore et dolore magna aliqua. Ut enim ad minim veniam, quis nostrud exercitation ullamco laboris nisi ut aliquip ex ea commodo consequat. Duis aute irure dolor in reprehenderit in voluptate velit esse cillum dolore eu fugiat nulla pariatur. Excepteur sint occaecat cupidatat non proident, sunt in culpa qui officia deserunt mollit anim id est laborum.

%Lorem ipsum dolor sit amet, consectetur adipisicing elit, sed do eiusmod tempor incididunt ut labore et dolore magna aliqua. Ut enim ad minim veniam, quis nostrud exercitation ullamco laboris nisi ut aliquip ex ea commodo consequat. Duis aute irure dolor in reprehenderit in voluptate velit esse cillum dolore eu fugiat nulla pariatur. Excepteur sint occaecat cupidatat non proident, sunt in culpa qui officia deserunt mollit anim id est laborum.

%Lorem ipsum dolor sit amet, consectetur adipisicing elit, sed do eiusmod tempor incididunt ut labore et dolore magna aliqua. Ut enim ad minim veniam, quis nostrud exercitation ullamco laboris nisi ut aliquip ex ea commodo consequat. Duis aute irure dolor in reprehenderit in voluptate velit esse cillum dolore eu fugiat nulla pariatur. Excepteur sint occaecat cupidatat non proident, sunt in culpa qui officia deserunt mollit anim id est laborum.

%\section*{Überschrift}

%Lorem ipsum dolor sit amet, consectetur adipisicing elit, sed do eiusmod tempor incididunt ut labore et dolore magna aliqua. Ut enim ad minim veniam, quis nostrud exercitation ullamco laboris nisi ut aliquip ex ea commodo consequat. Duis aute irure dolor in reprehenderit in voluptate velit esse cillum dolore eu fugiat nulla pariatur. Excepteur sint occaecat cupidatat non proident, sunt in culpa qui officia deserunt mollit anim id est laborum.

%Lorem ipsum dolor sit amet, consectetur adipisicing elit, sed do eiusmod tempor incididunt ut labore et dolore magna aliqua. Ut enim ad minim veniam, quis nostrud exercitation ullamco laboris nisi ut aliquip ex ea commodo consequat. Duis aute irure dolor in reprehenderit in voluptate velit esse cillum dolore eu fugiat nulla pariatur. Excepteur sint occaecat cupidatat non proident, sunt in culpa qui officia deserunt mollit anim id est laborum.

%Lorem ipsum dolor sit amet, consectetur adipisicing elit, sed do eiusmod tempor incididunt ut labore et dolore magna aliqua. Ut enim ad minim veniam, quis nostrud exercitation ullamco laboris nisi ut aliquip ex ea commodo consequat. Duis aute irure dolor in reprehenderit in voluptate velit esse cillum dolore eu fugiat nulla pariatur. Excepteur sint occaecat cupidatat non proident, sunt in culpa qui officia deserunt mollit anim id est laborum.

% --------------------------------------------------------------------------------
% ----- Todo list
% --------------------------------------------------------------------------------
\listoftodos
% \todototoc

\end{document}
