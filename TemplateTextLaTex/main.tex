%!TEX encoding = UTF-8 Unicode
% --------------------------------------------------------------------------------
\documentclass[
	fontsize=12pt,
	headings=small,
	parskip=half,           % Ersetzt manuelles Setzen von parskip/parindent.
	bibliography=totoc,
	numbers=noenddot,       % Entfernt den letzten Punkt der Kapitelnummern.
	open=any,               % Kapitel kann auf jeder Seite beginnen.
%	final                   % Entfernt alle todonotes und den Entwurfstempel.
]{scrreprt}
% --------------------------------------------------------------------------------
% Hinweis: Das Übersetzen dieser Datei funktioniert auch Online mit 
% https://www.overleaf.com. Hierzu müssen neben dieser Datei im gleichen 
% Verzeichnis die Dateien hinweiseabschlussarbeit.bib und stylesvs.tex liegen.
% --------------------------------------------------------------------------------
\input{stylesvs}

\title{Wirtschaftsinformatik und Nachhaltigkeit:\\
Anwendungsszenarien in der Transportwirtschaft}
\author{Valentina Ermisch, Lisa-Sophie Kaisik}
% \date{01.01.2015} % Für bestimmtes Datum diese Zeile aktivieren.

\begin{document}

\newpage
\thispagestyle{empty}
\null
\newpage

% --------------------------------------------------------------------------------
% ----- Muster für weitere Deckblätter siehe hinten
% --------------------------------------------------------------------------------
\begin{titlepage}
\mbox{\parbox[t][1.75cm][b]{2.2cm}{\uhhlogo}}
\begin{center}\Large
	\vfill Seminararbeit
	\vfill \makeatletter {\Large\textsf{\textbf{\@title}}\par} \makeatother
	\vfill vorgelegt von \par\bigskip
	\begin{tabu}{p{0.5\textwidth}p{0.5\textwidth}}
	\centering Valentina Ermisch     & \centering Lisa-Sophie Kaisik \\[1ex]
	\centering Matrikelnummer 7775958 & \centering Matrikelnummer 7726396 \\
	\centering Studiengang Betriebs- & \centering Studiengang Wirtschafts- \\
	\centering wirtschaftslehre & \centering informatik \\

	\end{tabu}
	\vfill MIN-Fakultät \par Fachbereich Wirtschaftsinformatik
	\vfill \makeatletter eingereicht am {\@date} \makeatother
	\vfill Betreuer: Prof. Dr. Markus Nüttgens
\end{center}
\ifoptionfinal{}{
\begin{tikzpicture}[remember picture, overlay]
	\node[draw, red, font=\ttfamily\bfseries\Large, xshift=30mm, yshift=238mm, rotate=340, text centered, text width=6cm, very thick, rounded corners=4mm] at (current page.south) {Entwurf vom \today};
\end{tikzpicture}
%\begin{tikzpicture}[overlay]
%	\node[draw, blue, font=\sffamily\Large, xshift=0mm, yshift=242mm, rotate=0, text centered, rounded corners=1mm] at (current page.south) {Muster des Deckblatts für Seminararbeiten};
%\end{tikzpicture}
}
\end{titlepage}

% --------------------------------------------------------------------------------
% ----- Ab hier folgt der Haupttext
% --------------------------------------------------------------------------------

%\chapter*{Aufgabenstellung}

%Soweit eine ausformulierte Aufgabenstellung mit der Betreuerin bzw. dem Betreuer vereinbart wurde, diese bitte hier einfügen.

%\chapter*{Vorwort, Zusammenfassung}

%Für die eilige Leserin bzw. den eiligen Leser sollen auf etwa einer halben, maximal einer Seite die wichtigsten Inhalte, Erkenntnisse, Neuerungen bzw. Ergebnisse der Arbeit beschrieben werden.

%Durch eine solche Zusammenfassung (im Engl. auch Abstract genannt) am Anfang der Arbeit wird die Arbeit deutlich aufgewertet. Hier sollte vermittelt werden, warum man die Arbeit lesen sollte.

\pagenumbering{roman}

\tableofcontents

\cleardoublepage
\pagenumbering{arabic}
\setcounter{page}{1}

\chapter{Einleitung}

\section{Motivation und Problemstellung}

Von Jahr zu Jahr wird es immer w\"armer, weshalb es wichtiger denn je wird, den Zielen des Pariser Klimaabkommens n\"aherzukommen. Ziel des Abkommens ist es, den globalen Temperaturanstieg auf unter 2\,\textdegree C zu begrenzen. Hierf\"ur ist es insbesondere notwendig, die CO$_2$-Emissionen zu reduzieren, die zu einem Gro\ss teil durch den Transportsektor verursacht werden. 

Neben den klimatischen Herausforderungen entwickelt sich auch die Gesellschaft stetig weiter. Die Kundenanforderungen steigen und die Komplexit\"at logistischer Prozesse nimmt zu. F\"ur Unternehmer stellt Nachhaltigkeit daher einen entscheidenden Faktor f\"ur die langfristige Wettbewerbsf\"ahigkeit dar. Sie stehen vor der Herausforderung, nachhaltige L\"osungen entlang logistischer Prozesse zu entwickeln. Dabei soll nicht nur effizient, sondern zugleich auch nachhaltig geplant und gearbeitet werden. 

Die Wirtschaftsinformatik bietet Unternehmern die M\"oglichkeit, durch erh\"ohte Transparenz und Vernetzung ihre logistischen Netzwerke zu optimieren und neue IT-Systeme nicht ausschlie\ss lich als Kostenfaktoren, sondern als Chancen f\"ur Wachstum, Effizienz und bessere Organisation zu begreifen. Daraus ergibt sich folgende zentrale Fragestellung: \glqq Wie kann die Wirtschaftsinformatik in der Transportwirtschaft dazu beitragen, \"okologische Nachhaltigkeit zu f\"ordern und gleichzeitig Effizienz sicherzustellen?\grqq


\section{Zielsetzung}

Ziel dieser Seminararbeit ist es, Unternehmern konkrete Anwendungsm\"oglichkeiten f\"ur den Einsatz IT-gest\"utzter Logistikl\"osungen aufzuzeigen. Dabei wird dargelegt, wie wirtschaftsinformatische Ans\"atze zur Weiterentwicklung logistischer Prozesse beitragen k\"onnen, um Effizienz, Kosteneffektivit\"at und Nachhaltigkeit zu steigern und den wachsenden Anforderungen des Transportsektors gerecht zu werden.


\section{Aufbau der Semianrarbeit}

In dieser Arbeit werden zun\"achst die grundlegenden Begriffe und Zusammenh\"ange der Transportwirtschaft, der Logistik sowie der Nachhaltigkeit erl\"autert. Im Kontext der Nachhaltigkeit wird dabei kurz auf die Agenda~2030 eingegangen. Der Nachhaltigkeitsbegriff wird anhand der drei S\"aulen --- \"okologisch, \"okonomisch und sozial --- dargestellt und im Zusammenhang mit der Transportwirtschaft eingeordnet. Dar\"uber hinaus werden die Begriffe Logistik, Smart Logistics und Gr\"une Logistik definiert, um ein einheitliches Begriffsverst\"andnis f\"ur die folgenden Kapitel sicherzustellen.

Darauf aufbauend wird die Rolle der Wirtschaftsinformatik als Schnittstelle zwischen Informationstechnologie, Logistik und Betriebswirtschaftslehre aufgezeigt. Sie bildet die technologische Grundlage f\"ur Konzepte wie Smart Logistics und Logistik~4.0. In Abschnitt~2.3 werden zentrale Konzepte und Technologien wie Enterprise Resource Planning (ERP), Transportation Management-Systeme (TMS), das Internet of Things (IoT), intelligente Transportsysteme (ITS), Big Data, Cloud Computing, k\"unstliche Intelligenz (KI) sowie Blockchain-Technologien n\"aher erl\"autert. Erg\"anzend werden Ans\"atze wie Green~IT, Green~IS und die sogenannte Twin Transformation thematisiert.

Im Hauptteil der Arbeit werden konkrete IT-gest\"utzte L\"osungsans\"atze zur Steigerung von Effizienz und Nachhaltigkeit in der Transportwirtschaft analysiert. Dazu z\"ahlen insbesondere die strukturierte Routen- und Tourenoptimierung durch TMS- und ERP-Systeme, der Einsatz intelligenter Transportsysteme zur Echtzeitsteuerung von Transportprozessen sowie plattformbasierte Kooperationsmodelle zur Reduktion von Leerfahrten.

Dar\"uber hinaus werden Blockchain-Technologien und Smart Contracts als Instrumente zur Erh\"ohung von Transparenz, Sicherheit und Nachhaltigkeit in Lieferketten betrachtet. Abschlie\ss end wird der Einsatz von Predictive Logistics und k\"unstlicher Intelligenz diskutiert, die durch vorausschauende Planung, datenbasierte Prognosen und teilautonome Entscheidungen einen wesentlichen Beitrag zur ressourceneffizienten und nachhaltigen Gestaltung des Transportsektors leisten k\"onnen.

Zum Abschluss werden die wichtigsten Ergebnisse im Fazit zusammengefasst. Die zentrale Frage wird beantwortet und es werden die neuen Folgen, wie ein hoher Energieverbrauch der IT-Systeme diskutiert. Es gibt einen Ausblick und Hinweise auf zukünftige Entwicklungen. 

\chapter{Grundlagen}


\section{Transportwirtschaft}

Die Transportwirtschaft umfasst alle gesellschaftlichen, wirtschaftlichen und technischen Institutionen, die Einfluss auf Transportprozesse haben. Im Kern beinhaltet die Transportwirtschaft den kommerziellen Transport von G\"utern und Personen mithilfe von Verkehrstr\"agern. Verkehrstr\"ager lassen sich in drei wesentliche Gruppen unterteilen. Landverkehr mit Autos und Eisenbahn, Seeverkehr mit der Schifffahrt sowie Luftverkehr mit Flugzeugen. Es handelt sich hierbei nicht nur um den G\"utertransport, sondern auch den Passagiertransport, welcher auch ein wichtiger Teil der Transportwirtschaft ist.

Die Logistik umfasst alle Prozesse zur Raum\"uberbr\"uckung von Informationen und G\"utern. Vereinfacht gesagt, ist es die Bef\"orderung von einem Gut von A nach B. Dabei setzt diese auf die \glqq 6 R's\grqq: die richtige Menge, der richtigen Objekte, zur richtigen Zeit, mit der richtigen Information, den richtigen Kosten und am richtigen Ort bereitzustellen. (econstor.eu)

Smart Logistik ist die Optimierung logistischer Prozesse. Es werden Smart Technologien implementiert, die Basistechnologien vernetzen. Sie erm\"oglicht in Echtzeit \"uber das Internet intelligente Entscheidungen, Optimierungen und teilautonomie Steuerungen vorzunehmen. Die Grundlage f\"ur smarte Logistik sind K\"unstliche Intelligenz (KI), Automatisierung, dezentrale Datenverarbeitungen und das Internet of Things (IoT). Damit kann eine Umgebung geschaffen werden, in der Objekte \"uber das Internet miteinander kommunizieren und Daten ausgetauscht werden.
(\,Smarte Logistik: Hebel der Digitalisierung -- Die BVL: Das Logistik-Netzwerk f\"ur Fach- und F\"uhrungskr\"afte)

Eine weitere Variante der Logistik ist die Gr\"une Logistik, welche in mehreren Bereichen der Logistik vertreten ist. Entstanden ist die gr\"une Logistik mit den Zielen, der Nachhaltigkeit und des Umweltschutzes. Logistikprozesse wie Transport und Lagerung sollen m\"oglichst geringen CO$_2$-Aussto\ss\ aufweisen, dabei sollten diese auf die 6rs der Logistik angewendet werden und Nachhaltigkeit mit Effizienz verbinden. (nachhaltigkeit-wirtschaft.de)


\section{Nachhaltigkeit mit Bezug auf Transportwirtschaft}

Der Begriff der Nachhaltigkeit besteht aus drei Hauptkomponenten, sogenannten drei S\"aulen: der \"okologischen, der \"okonomischen und der sozialen Nachhaltigkeit. Die \"okologische Nachhaltigkeit beschreibt den Schutz des Klimas, der Biodiversit\"at und die bedachte Nutzung der Ressourcen. Hierbei gilt f\"ur Unternehmen ein ressourcenschonendes Wirtschaften, damit die Umweltbelastung minimiert wird. Die \"okonomische Komponente zeichnet den bedachten Konsum von G\"utern aus. F\"ur Unternehmen gilt, Profit langfristig m\"oglichst nachhaltig zu generieren. Die soziale Nachhaltigkeit setzt auf Bildung, Gesundheit und Chancengleichheit f\"ur jeden. Es wird darauf abgezielt, ein gerechtes Umfeld insbesondere f\"ur benachteiligte Menschen zu schaffen (Greenpeace). F\"ur die Transportwirtschaft bedeutet der \"okologische Aspekt, die Emissionen sowie den Energie- und Fl\"achenverbrauch von Transportketten zu verringern (L-0016436356-pdf.pdf). \"Okonomische Nachhaltigkeit schafft die Transportwirtschaft mittels effizienter Netzwerke und klimafreundlicher Technologien. Die soziale Nachhaltigkeit schafft der Transportsektor durch ausgewogene Arbeitszeiten, Vermeidung von L\"arm und transparente Einbindung von Stakeholdern in Transformationsprozesse.

Die Agenda~2030 beinhaltet 17 globale Zukunftsziele f\"ur nachhaltige Entwicklung. Verabschiedet wurden die 17 Sustainable Development Goals (kurz SDGs) am 25.~September~2015 und traten am 1.~Januar~2016 in Kraft. Die Agenda~2030 erweiterte somit die Ziele des Vorg\"angerprogramms Agenda~21. Alle 193 UN-Mitgliedstaaten verabschiedeten diese Ziele, wobei es sich gesetzlich lediglich um eine freiwillige Selbstverpflichtung der Staaten handelt (lpb-bw). Das Ziel ist vor allem


\section{Wirtschaftsinformatik und Smart Logistics}

Wirtschaftsinformatik besch\"aftigt sich mit der Planung, Entwicklung und Anwendung von Informations- und Kommunikationssystemen in Unternehmen. Ihr Ziel ist es, Gesch\"aftsprozesse durch den Einsatz von Informationstechnologie effizient zu unterst\"utzen. Sie bildet die Schnittstelle zwischen Informatik und Betriebswirtschaftslehre und verbindet technische L\"osungen mit wirtschaftlichen Anforderungen. Dabei untersucht die Wirtschaftsinformatik, wie betriebliche Abl\"aufe mithilfe von IT-Systemen gestaltet und verbessert werden k\"onnen. Eine weitere Aufgabe der Wirtschaftsinformatik ist die Analyse und Gestaltung von Informationssystemen sowie das Management von Informationen und IT-Ressourcen. Die Wirtschaftsinformatik betrachtet technische, wirtschaftliche und organisatorische Aspekte beim Einsatz von Informationssystemen (Informatik -- Definition \textbar\ Gabler Wirtschaftslexikon).

Smart Logistics ist eine neue Entwicklung der einfachen Logistik, die durch den Einsatz von Technologien die Effizienz, Kosteneffektivit\"at und Nachhaltigkeit maximieren soll (Developing Smart Logistics for.pdf). Es geht darum, intelligente Vernetzungen von autonomen und selbststeuernden Ressourcen zu entwickeln. Die Logistik~4.0 beschreibt die Integration der Prinzipien der Industrie~4.0 in den Logistiksektor, bei der physische und virtuelle Welten zusammentreffen (Developing Smart Logistics for.pdf, IT-gest\"utzte-Logistik\_Iris Hausladen.pdf).

Die Wirtschaftsinformatik stellt eine Menge an Basistechnologien zur Verf\"ugung, die der digitalen Transformation helfen. Enterprise Resource Planning (ERP) ist eine modulare Softwarel\"osung, die eine unternehmensweite Integration zwischen Personal, Finanzen und Werkstoffen erm\"oglichen soll [Hau20]. Erweitert wird ERP durch Supply Chain Management (SCM) und Customer Relationship Management (CRM), die auch unternehmens\"ubergreifende Prozesse abbilden k\"onnen (IT-gest\"utzte-Logistik\_Iris Hausladen.pdf).

Transport Management Systeme (TMS) sind spezialisierte Anwendungen, die sich gemeinsam mit ERP-Anwendungen auf die Managementfunktionen der Unternehmen konzentrieren. Es geht um die Planung, Durchf\"uhrung und \"Uberwachung von Transportprozessen. Hauptaufgaben sind Routen- und Tourenplanung sowie Tracking und Tracing. Tracking beschreibt die Erfassung und Analyse von Informationen. Tracing ist die Verfolgung der Waren, sowohl vor, w\"ahrend als auch nach der Lieferung (IT-gest\"utzte-Logistik\_Iris Hausladen.pdf).

Das Internet of Things (IoT) beschreibt ein Netzwerk an physischen Objekten, wie beispielsweise Fahrzeugen oder Containern. Diese sind mit Sensoren und Software ausgestattet, die Daten direkt \"uber das Internet austauschen (Developing Smart Logistics for.pdf, IT-gest\"utzte-Logistik\_Iris Hausladen.pdf).

Intelligente Transportsysteme (ITS) sind Systeme, bei denen modernste Informations- und Kommunikationstechnologie (IKT) in Transportnetze eingebunden werden. Sie beinhalten die Infrastruktur, Fahrzeuge, Nutzer und das Verkehrsmanagement. Am einfachsten sind ITS anhand von intelligenten Ampelsteuerungen vorstellbar (Impact of ITS Applications on Green Logistics and Customer Service.pdf).

Big Data bezeichnet das Verarbeiten gro\ss er und komplexer Datenmengen. Diese k\"onnen mittels Data Mining analysiert werden (IT-gest\"utzte-Logistik\_Iris Hausladen.pdf, 140714\_Positionspapier\_BigData4 (3).pdf). Clouds erm\"oglichen flexiblen Zugriff auf IT-Leistungen \"uber das Internet, ohne eigene Server. Als Blockchain werden dezentrale, f\"alschungssichere Datenbanken bezeichnet. Diese erm\"oglichen einen sicheren Austausch von Informationen ohne weitere Intermedi\"are (Hausladen.pdf, fir\_Janssen\_et\_al\_Abschlussbericht.pdf).

KI und Logistik: K\"unstliche Intelligenz (KI), im Englischen auch Artificial Intelligence (AI) genannt, l\"asst sich als Algorithmen beschreiben, die menschliche kognitive F\"ahigkeiten replizieren. In Bezug auf Computer sind Algorithmen Arbeitsanweisungen. Diese Anweisungen werden von Menschen entwickelt und programmiert. Sie legen fest, welche Schritte ein Computer ausf\"uhren soll und in welcher Reihenfolge. Algorithmen werden genutzt, um Daten nach festen Regeln zu analysieren. Das bedeutet, dass Daten miteinander verkn\"upft und verarbeitet werden. Dabei entstehen neue Daten, die f\"ur Menschen verst\"andlich und f\"ur bestimmte Personen besonders interessant sind. Diese neuen Daten bezeichnet man als Informationen. Algorithmen sind in der Lage, sehr gro\ss e Datenmengen in kurzer Zeit zu vergleichen und daraus ein bestimmtes Ergebnis zu ermitteln (Algorithmus \textbar\ bpb.de).

KI analysiert gro\ss e Datens\"atze. Die Logistik mit ihren weit verzweigten Netzwerken ist ein ideales Anwendungsfeld f\"ur K\"unstliche Intelligenz. Mit intelligent ausgewerteten Daten lassen sich beispielsweise zuk\"unftige Produktions- und Transportmengen prognostizieren. So k\"onnen Unternehmen ihre Ressourcen effizienter einsetzen. Solche Aufgaben werden zunehmend von selbstlernenden digitalen Systemen unterst\"utzt oder \"ubernommen. KI-Modelle k\"onnen gro\ss e, heterogene Datenmengen filtern und in diesen Muster erkennen. Aus diesen Mustern k\"onnen autonome Entscheidungen f\"ur die Routenplanung oder Bedarfsprognosen abgeleitet werden. Predictive Logistics nutzt ebenfalls die F\"ahigkeiten der KI, um St\"orungen vorauszuberechnen und Fehler zu vermeiden (Sustainable Freight Transport in Support of the 2030 Agenda for Sustainable Development.pdf, 1-s2.0-S2949899624000042-main.pdf).

Die Green~IT wird auch als \glqq Greening of IT\grqq\ bezeichnet und setzt ihren Fokus auf die umweltfreundliche Gestaltung der Hardware selbst, beispielsweise auf energieeffiziente Rechenzentren oder ressourcenschonende Hardware-Entsorgung (1-s2.0-S0268401224000021-main.pdf). Green~IS, auch \glqq Greening by IT\grqq, ist der Einsatz von Informationssystemen, um die Nachhaltigkeit von Gesch\"aftsprozessen zu steigern. Das sind die L\"osungen einiger Probleme der Transportwirtschaft, wie die Reduktion von Leerfahrten durch digitale Plattformen (L-0016436356-pdf.pdf).

Als Twin Transformation wird das Voranbringen der Digitalisierung mit dem gleichzeitigen Voranbringen der Nachhaltigkeit definiert. Digitalisierung und Nachhaltigkeit agieren hierbei gemeinsam, sodass Digitalisierung eine Teill\"osung f\"ur die Nachhaltigkeitsziele ist und die Nachhaltigkeit der digitalen Transformation einen Zweck gibt (ey-studie-digital-und-nachhaltig-die-zukunft-sichern-februar-2023.pdf).


\chapter{Anwendungsszenarien für IT-gestützte Lösungsansätze}

\section{Transportation Management Systeme und Enterprise Resource Planning}

In der modernen Transportwirtschaft bilden Enterprise-Resource-Planning (ERP)- und Transport-Management-Systeme (TMS) die Grundlage für die Koordination physischer Warenströme und begleitender Datenflüsse. \cite{Hausladen2020} Für Geschäftsführer sind diese Systeme nicht nur operative Werkzeuge, sondern bieten die Möglichkeit der „Twin Transformation“. Sie verknüpfen ökonomische Effizienzsteigerung direkt mit ökologischen Nachhaltigkeitszielen. \cite{EY2023} Durch die Integration lässt sich die Komplexität globaler Netzwerke reduzieren, was die Basis für eine nachhaltige Standortsicherung und Wettbewerbsfähigkeit bildet. [Kad+24]

Die Architektur moderner ERP-Systeme ist modular aufgebaut und zielt auf die unternehmensweite Integration aller Ressourcen wie Personal, Finanzen und Material ab. [Hau20] Das TMS übernimmt innerhalb dieser Struktur die Planung, Durchführung und Überwachung von Transportprozessen. [BL14]
Die Integration von Auftrags- und Fuhrparkmanagement schafft durch die Verknüpfung von Auftragsdaten aus dem ERP mit den operativen Kapazitäten des TMS eine medienbruchfreie Prozesskette von der Bestellung bis zur Auslieferung. [Hau20] Das zentrale Datenmanagement dient als Single Source of Truth, konsolidiert heterogene Datenbestände, vermeidet Redundanzen und sichert hohe Datenqualität für die Entscheidungsunterstützung. [Hau20] Moderne Architekturen nutzen zudem Cloud-Lösungen und standardisierte Schnittstellen (EDI/XML), um externe Partner und mobile Endgeräte in Echtzeit zu integrieren.

Algorithmische Verfahren in TMS-Lösungen sind elementar, um ökonomische und ökologische Ziele zu verbinden. [SR22] Da der Transport rund 90\% der logistikbedingten Treibhausgasemissionen verursacht, führt jede Routenoptimierung zu einer messbaren Umweltentlastung. [Dec21]
Mathematische Algorithmen zur Ressourcenproduktivität erhöhen die Auslastung von Transportmitteln und reduzieren gefahrene Kilometer. Praxisbeispiele wie bei Bosch zeigen Einsparungen von bis zu 15\% der Logistikkosten. [SR22] Die Senkung des Kraftstoffverbrauchs erfolgt durch optimierte Tourenplanung und die Vermeidung von Leerlauf. Schulungen und intelligente Systeme können den Dieselverbrauch um bis zu 5 Liter pro 100 km senken. [Dec21] IT-gestützte Planung beim Modal Split ermöglicht die Kombination verschiedener Verkehrsträger (Schiene, Straße, Wasser) und verringert so die Umweltauswirkungen insgesamt. [Dec21]
Der Einsatz von Künstlicher Intelligenz (KI) und Advanced Analytics hebt die Netzwerkoptimierung auf eine vorausschauende Ebene. [RD22] Während klassische Systeme reaktiv planen, ermöglichen datenbasierte Prognosen proaktives Handeln. [BL14]
KI-gestützte Analysen historischer Daten in der prognosebasierten Kapazitätsplanung helfen, Nachfrageschwankungen früh zu erkennen und Kapazitäten gezielt bereitzustellen. [ey-studie-digital-und-nachhaltig-die-zukunft-sichern-februar-2023.pdf] Intelligente Logistikplattformen zur Vermeidung von Leerfahrten bündeln Fracht- und Laderaumangebote branchenübergreifend und senken Leerfahrten im Fernverkehr auf unter 10\%. [Dec21] Die Echtzeit-Entscheidungsunterstützung durch die Verarbeitung von Sensordaten (IoT) ermöglicht automatische Routenanpassungen bei Störungen, was die Resilienz der Lieferketten stärkt und unnötige Umwege vermeidet. [Kc24]

\section{Intelligente Transportsysteme und Internet of Things}

Die Logistik ist zunehmend auf Informations- und Kommunikationstechnologien (IKT) angewiesen, um wachsende Warenströme und die dazugehörigen Informationsflüsse effizient zu steuern. [Hau20] Intelligente Transportsysteme (ITS) stellen dabei eine zentrale Verbindung zwischen Informatik und Telekommunikation her und erhöhen die Leistungsfähigkeit, Sicherheit und Wirtschaftlichkeit von Transportprozessen. [AH06, Hau20] Für Geschäftsführer eines Transportunternehmens können Investitionen in Internet-of-Things (IoT)-Lösungen nicht nur Kostenvorteile bringen, sondern auch neue strategische Entwicklungsmöglichkeiten eröffnen.

IoT beschreibt ein Netzwerk physischer Objekte mit Sensoren, Software und Internetanbindung, das Daten erfasst, austauscht und mit anderen Systemen kommuniziert. Es trägt zur Ressourcenproduktivität bei, indem es Transparenz schafft und eine Echtzeitsteuerung ermöglicht. Dadurch entstehen erhebliche Effizienzgewinne bei gleichzeitig geringeren ökologischen Fußabdruck des Unternehmens. [SR22]

Durch die Vernetzung von Fahrzeugen über IoT-Sensoren und GPS können Disponenten Transportwege dynamisch an die aktuelle Verkehrslage anpassen. [Hau20, Kc24] Wirtschaftsinformatische Algorithmen vermeiden Staus und Leerfahrten und senken so Fahrzeit und Energieverbrauch um bis zu 40–70\%. [Kad+24] Ein Praxisbeispiel ist die unternehmensübergreifende Netzwerkoptimierung großer Konzerne, die durch algorithmische Planung bis zu 15\% der Logistikkosten bei deutlich geringerem CO$_2$-Ausstoß reduziert. [BB12, SR22]

IoT-basierte Container nutzen Sensoren und Telematik, um Zustandsdaten wie Temperatur, Feuchtigkeit und Erschütterungen in Echtzeit zu überwachen. [RD22] Besonders in der Kühlkettenlogistik – etwa in der Lebensmittel- oder Pharmabranche – reduziert dies Warenverluste durch Verderb und steigert die Energieeffizienz von Kühlanlagen. [Hau20] Gleichzeitig wird die lückenlose Rückverfolgung zum Standard: Tracking und Tracing sichern die Produktqualität und helfen, regulatorische Anforderungen zuverlässig zu erfüllen. [SJR23]

ITS-Anwendungen im Fahrzeugmanagement unterstützen eine energieeffiziente Fahrweise (Eco Driving) durch die Analyse von Motordaten und die Anzeige optimaler Geschwindigkeiten. [IKad+24] In der Praxis senken digitale Fahrerschulungen den Kraftstoffverbrauch langfristig um bis zu 5 Liter Diesel pro 100 km. [Dec21] Solche Systeme reduzieren Schadstoff- und Lärmemissionen direkt an der Quelle und leisten damit einen messbaren Beitrag zur Umweltleistung. [SR22]
Zukunftsorientierte Konzepte wie das Truck Platooning vernetzen mehrere Lkw zu einem digitalen Konvoi. [2375Mobilizing Sustainable Transport.pdf] Die Fahrzeuge halten automatisch den optimalen Abstand, wodurch sich der Luftwiderstand verringert und Kraftstoff eingespart wird. Das trägt direkt zur Dekarbonisierung des Straßengüterverkehrs bei. [Transportation Report 2021FullReportDigital.pdf] Pilotprojekte, etwa in Singapur, zeigen, dass diese Systeme zugleich dem Fachkräftemangel entgegenwirken und die Verkehrssicherheit erhöhen, da menschliche Fehler reduziert werden.

IoT und ITS ermöglichen den Übergang zu einer intelligenten, datenbasierten Logistik, in der nahezu alle Prozesse in Echtzeit steuerbar sind. Dadurch entsteht die Grundlage für eine nachhaltige Transformation und langfristige Wettbewerbsfähigkeit von Transportunternehmen. [RD22, SR22]

\section{Plattformökonomie und datengetriebene Kooperationsmodelle}

Neben TMS und ERP ermöglichen auch Plattformmodelle die Umsetzung der “Twin-Transformation” in der von Volatilität und hohen Kostendruck geprägten Logistikwelt. [Transportation Report 2021FullReportDigital.pdf] \cite{EY2023} Für Geschäftsführer stellen sie ein zentrales Steuerungsinstrument dar, um die Fragmentierung des Transportsektors zu überwinden und Managemententscheidungen datengestützt und proaktiv zu treffen. [Transportation Report 2021FullReportDigital.pdf] Digitale Plattformen transformieren die Logistik von einer Dienstleistung zu einem wertschöpfenden, integrierten Netzwerkmanagement. [Hau20]
Digitale Frachtbörsen fungieren als virtuelle Marktplätze, die Angebot und Nachfrage für Transportkapazitäten in Echtzeit zusammenführen. [Hau20] Spotmärkte decken kurzfristige Bedarfe ab, während Kontaktmärkte langfristige Kooperationen ermöglichen. Beide Ansätze verbessern die Ressourcennutzung und lösen starre zweiseitige Verträge ab. [Hau20]
Durch intelligente Algorithmen werden Ladungen unternehmensübergreifend gebündelt, was im Straßengüterverkehr die Markttransparenz erhöht und den Zugang zu verfügbaren Kapazitäten erleichtert. [Kad+24, Dec21] 
Praxisbeispiele wie Teleroute oder TRANS.eu zeigen, dass Transportunternehmen so ihre operative Flexibilität steigern und Laderaum europaweit effizienter nutzen können. [Hau20]

Die Leistungsfähigkeit moderner Kooperationsmodelle hängt von der Integration unterschiedlicher Systeme wie ERP, TMS und Telematik ab. Cloud-Plattformen wie GT Nexus oder AX4 schaffen eine Single Source of Truth, indem sie Statusmeldungen und Dokumente über Unternehmensgrenzen hinweg in einer einheitlichen Datenumgebung zusammenführen. [Hau20]
Einheitliche Datenstandards wie EDI oder XML sind dabei entscheidend, um Medienbrüche zu vermeiden und eine vollständige Transparenz der Warenströme sicherzustellen. Diese Plattformen wirken wie ein digitales Nervensystem, das den physischen Transportfluss nahtlos mit dem Informationsfluss verbindet. [1-s2.0-S0268401224000021-main.pdf, Hau20]

Plattformgestützte Kooperationen erhöhen die Ressourcenproduktivität messbar. Durch die Reduktion von Leerfahrten im Fernverkehr auf unter 10\% können hohe Effizienzgewinne erzielt werden. [Dec21] Unternehmen wie Bosch zeigen, dass sich durch algorithmische Netzwerkplanung bis zu 15\% der Logistikkosten einsparen lassen. Gleichzeitig sinken Energieverbrauch und Treibhausgasemissionen um 30 bis 50\%. [Kad+24, SR22]
Die durch Plattformen gewonnene Transparenz ermöglicht eine vorausschauende („Predictive“) Logistik. Auf Basis verifizierter Daten lassen sich Fahrten präziser planen und Transportwege verkürzen. Das senkt den ökologischen Fußabdruck und stärkt zugleich die Wettbewerbsfähigkeit von Transportunternehmen. [UNC18, SR22]

\section{Blockchain Technologien}

Logistik und Supply Chain Management (SCM) hängen heutzutage stark von der Fähigkeit ab, fälschungssichere Informationen über die gesamte Lieferkette hinweg auszutauschen. [SJR23] Für Geschäftsführer von Transportunternehmen rückt das Thema Transparenz in den Fokus, da Kunden und Stakeholder zunehmend verifizierte Nachweise über Produktionsbedingungen, Herkunft und ökologische Auswirkungen von Produkten fordern. Die Blockchain dient dabei als technologische Grundlage, um von einer reaktiven Informationsweitergabe zu einer proaktiven Steuerung dynamischer Wertschöpfungsketten überzugehen. [SJR23] Man kann sich die Blockchain in einer Lieferkette wie ein gemeinsames Kassenbuch vorstellen, in dem jede Partei ihre Einträge mit unlöschbarer Tinte vornimmt. Da alle Teilnehmer identische Kopien besitzen, erkennt das System sofort, wenn jemand eine Seite manipulieren will. So entsteht Vertrauen in ein globales Netzwerk, ohne dass eine zentrale Instanz nötig ist.

Die Blockchain-Technologie basiert auf einer dezentralen Datenhaltung, bei der alle Teilnehmer des Netzwerks eine Kopie des Transaktionsverzeichnisses speichern. [Hau20, SJR23] Dies eliminiert den Bedarf an zentralen Intermediären und erhöht die Datensicherheit, da Informationen bei Systemausfällen an anderen Knotenpunkten bestehen bleiben. [SJR23] Die Unveränderbarkeit der Daten wird durch kryptografische Hashfunktionen sichergestellt. Sobald ein Datenblock erstellt ist, kann er nicht mehr gelöscht oder verändert werden, ohne dass das gesamte Netzwerk die Manipulation erkennt. [SJR23]
Ein zentraler betriebswirtschaftlicher Hebel liegt in Smart Contracts. Dabei handelt es sich um automatisierte Wenn‑Dann‑Beziehungen im Code, die Geschäftsprozesse wie Zahlungen bei Wareneingang oder Zollmeldungen selbstständig ausführen können – ohne manuellen Eingriff. [Hau20, SJR23]

Blockchain‑Anwendungen ermöglichen eine lückenlose Produktrückverfolgung über den gesamten Lebenszyklus hinweg. [SJR23] Im Kontext der Nachhaltigkeit erlaubt die Technologie eine manipulationssichere Erfassung und Verteilung von Primärdaten zu CO$_2$‑Emissionen entlang der gesamten Transportkette. [SJR23] Durch automatisierte Verifizierungen („Verifiable Proofs“) wird der administrative Aufwand für die CO$_2$-Bilanzierung deutlich reduziert, während die Glaubwürdigkeit gegenüber Kunden steigt. [SJR23]


Für kleine und mittlere Unternehmen entstehen durch mehr Datensouveränität und den Verzicht auf Intermediäre deutliche Kostenvorteile. [SJR23]
Die Möglichkeit, Nachhaltigkeitsnachweise – etwa für das Lieferkettensorgfaltspflichtengesetz – rechtskonform bereitzustellen, schafft einen strategischen Wettbewerbsvorteil. [SJR23]
Zu den Barrieren zählen ein hoher Initialaufwand und eine geringe Datenfreigabebereitschaft der Partner, die Wettbewerbsbedenken haben. [SJR23] Zudem erfordert die dezentrale Speicherung großer Datenmengen erhebliche Serverkapazitäten, was zu höherem Energieverbrauch führen kann. [SJR23]
Technologische Risiken wie Cyberangriffe auf Schnittstellen müssen über ein integriertes IT‑Sicherheitsmanagement gesteuert werden. [SJR23]

Beispielhaft haben Maersk und IBM eine offene Digitalplattform entwickelt, die grenzüberschreitende Warenflüsse transparent abbildet und papierlose Prozesse ermöglicht. Hyundai Merchant Marine testete ein Blockchain‑basiertes System zur Optimierung der Containerlogistik und zur Überwachung von Lieferketten in Echtzeit.
Projekte wie SiLKe nutzen die Blockchain zur Rückverfolgung von Lebensmitteln, um im Krisenfall – etwa bei Rückrufen – schneller und gezielter reagieren zu können. [SJR23]

\section{Predictive Logistics und Künstliche Intelligenz}

In einem volatilen Marktumfeld entscheidet die Leistungsfähigkeit der Logistik über die Stabilität globaler Lieferketten. Versagt sie, drohen Unterbrechungen in der gesamten Wertschöpfungskette. [1-s2.0-S2949899624000042-main.pdf] Für Geschäftsführer ist es wichtig, den Schritt von reaktiver Hektik zu vorausschauender Steuerung zu vollziehen. [BL14, Hau20]
Durch die Verknüpfung interner ERP-Daten mit externen Informationsquellen lassen sich lokale Effizienzgrenzen überwinden. Auf diese Weise können bis zu 75\% der strukturellen Verbesserungspotenziale im Netzwerk genutzt werden. [BL14] Man kann sich Predictive Logistics wie einen erfahrenen Schachgroßmeister vorstellen. Während ein traditionelles System nur auf den aktuellen Zug reagiert, berechnet die KI hunderte Züge im Voraus. Sie erkennt drohende Hindernisse, wägt Optionen ab und wählt den Weg, der Ressourcen spart und das Ziel mit minimalem Risiko erreicht.

Predictive Logistics nutzt Algorithmen, um auf Basis historischer Transaktionen und aktueller Echtzeitdaten das Verhalten von Kunden, Fahrern und Märkten vorherzusagen. [UNC18]
Im Straßengüterverkehr ermöglichen IoT-Sensoren und Telematiksysteme die ständige Überwachung von Fahrzeugzuständen wie Standort oder Temperatur. Externe Daten wie Wetterprognosen oder Verkehrsanalysen fließen in die Modelle ein, damit das System Störungen vorausschauend erkennt. [UNC18]
Diese vorausschauende Planung erlaubt eine präzisere Bestandsführung und verbessert die Planbarkeit der Warenströme erheblich. [BL14, Hau20]

Künstliche Intelligenz (KI), insbesondere Maschinelles Lernen (ML), identifiziert in großen Datenmengen Muster und Zusammenhänge und entwickelt darauf aufbauend autonome Handlungsstrategien. [Hau20] Neuronale Netze und Deep‑Learning‑Modelle erkennen komplexe Nachfragemuster oder Fahrverhalten und bilden so einen digitalen Schatten der Logistikprozesse, der Echtzeit-Entscheidungen ermöglicht. [Hau20] In TMS-Umgebungen berechnen selbstlernende genetische Algorithmen unter Berücksichtigung von Zeit, Kosten und Emissionen die effizienteste Route. [Role and Applications of Advanced Digital Technologies in Achieving Sustainability in Multimodal Logistics Operations.pdf] KI-gestützte Systeme führen Aufgaben wie Routenplanung oder Bestellprognosen zunehmend autonom aus. [Developing Smart Logistics for.pdf]

Der Einsatz von KI ist ein wesentlicher Hebel zur Steigerung der Ressourcenproduktivität. [SR22] KI-Analysen ermöglichen die Vermeidung von Leerfahrten durch prädiktive Fahrzeugdisposition und gemeinsame Auslastung über Unternehmensgrenzen hinweg. Dadurch können Leerfahrten im Fernverkehr auf unter 10\% sinken. [UNC18, Dec21] Intelligente Verkehrssteuerung und Fahrerassistenzsysteme reduzieren Wartezeiten, Kraftstoffverbrauch und Emissionen um 30 bis 50\%. [Kad+24f, Kc24] Vorausschauende Wartung (Predictive Maintenance) nutzt Sensordaten, um Ausfälle früh zu erkennen. Dadurch werden Stillstände vermieden und die Nutzungsdauer von Fahrzeugen und Anlagen verlängert. \cite{EY2023}

KI bietet vor allem kleinen und mittelgroßen Transportunternehmen die Chance, Kosten durch Automatisierung zu senken und die Servicequalität durch präzise Lieferzeitprognosen zu verbessern. [Kc24, SJR23]

Zu den Herausforderungen zählen mangelnde Datenqualität und hohe Anfangsinvestitionen für IT‑Infrastruktur und Fachpersonal. [1-s2.0-S0268401224000021-main.pdf, Kc24] Ein Rebound‑Effekt kann auftreten, wenn Effizienzgewinne durch steigende Transportnachfrage kompensiert werden. \cite{EY2023} [SR22] Zudem ist die Akzeptanz im Management oft gering, weil der Return on Investment (ROI) schwer messbar ist. [1-s2.0-S2949899624000042-main.pdf, Kc24]


\chapter{Fazit}

Die Wirtschaftsinformatik schafft es, die Transportwirtschaft beim Umweltschutz zu unterstützen. Intelligente Systeme planen Routen viel genauer und effizienter und können damit Treibstoff einsparen. Spezielle Programme für das Transport-Management senken gleichzeitig die Kosten für den CO\textsubscript{2}-Ausstoß. Digitale Plattformen sorgen durch Vernetzung dafür, dass deutlich weniger Leerfahrten zustande kommen. Letztlich können Probleme durch Künstliche Intelligenz frühzeitig erkannt und vermieden werden. 

Die zentrale Leitfrage lässt sich klar beantworten. Die Wirtschaftsinformatik schafft mehr Transparenz und vernetzt alle Bereiche einer Lieferkette miteinander. Sie kann Unternehmen die Weiterentwicklung ermöglichen und damit Geld sparen und die Natur schonen. Arbeitszeiten und Fahrzeuge können durch genauere Daten optimal genutzt werden.

Aus den vorgestellten Techniken entstehen dennoch ein paar neue Probleme. Die Einführung dieser Systeme kostet zu Beginn viel Geld. Kleine Firmen können die Implementierung meist nicht tragen. Die genutzten Daten in den Systemen müssen sorgfältig und korrekt aufbereitet sein, damit die Programme brauchbare Ergebnisse erzielen. Ebenfalls führt die Nutzung von Computersystemen selbst zu viel Energie. Große Rechenzentren brauchen viel Strom und müssen gekühlt werden. Außerdem bedeuten effizientere Transporte auch eine noch höhere Nachfrage, die wiederum zu noch mehr Verkehr führen kann. Die Belastung der Umwelt könnte durch hohen Einsatz von moderner Technik also ebenfalls steigen. Weiterhin gilt es für die Unternehmen, die Sicherheit von Daten vor Angriffen durch Hacker zu schützen. 

Mit unserem Umfang wurden in dieser Arbeit nicht alle Details der Methoden untersucht. Unser Fokus lag auf den Techniken und ihrer ökologischen Wirkung.

\chapter{Ausblick}

In der Zukunft wird die Sicherheit der IT-Systeme eine der größten Rollen spielen. Unternehmen sollten von Beginn an auf Energieverbrauch und umweltfreundliche Entsorgung ihrer Hardware achten. 

Dennoch bietet die Digitalisierung eine der wichtigsten Rollen für die Weiterentwicklung umweltfreundlicher Logistik. Durch Künstliche Intelligenz kann in Zukunft noch viel weiter als bis zum nächsten Schritt der Lieferkette gedacht werden und im Voraus geplant werden.



% --------------------------------------------------------------------------------
% ----- Literaturverzeichnis
% --------------------------------------------------------------------------------
\begin{raggedright} % raggedright schaltet den Blocksatz ab und erzeugt ein stimmigeres Schriftbild im Literaturverzeichnis.
  \printbibliography % alphabetic ist definiert unter biblatex in style.svs
  \label{sec:literaturverzeichnis}
\end{raggedright}

% --------------------------------------------------------------------------------
% ----- Anhang
% --------------------------------------------------------------------------------
\appendix
\setcounter{figure}{0}
\renewcommand\thetable{A.\arabic{figure}}
\setcounter{table}{0}
\renewcommand\thetable{A.\arabic{table}}

% --------------------------------------------------------------------------------
% ----- Eidesstattliche Versicherung
% --------------------------------------------------------------------------------
\chapter*{Eidesstattliche Versicherung}
\vspace{1cm}

Ich erkläre eidesstattlich, dass ich die Arbeit selbständig angefertigt, keine anderen als die angegebenen Hilfsmittel benutzt und alle aus ungedruckten Quellen, gedruckter Literatur oder aus dem Internet im Wortlaut oder im wesentlichen Inhalt übernommenen Formulierungen und Konzepte gemäß den Richtlinien wissenschaftlicher Arbeiten zitiert, durch Fußnoten gekennzeichnet bzw. mit genauer Quellenangabe kenntlich gemacht habe.

Ich versichere, dass auch im Anwendungsfall von generativer Künstlicher Intelligenz (genKI) meine eigene schöpferische Leistung der erhebliche Anteil in dieser Seminararbeit ist und ich die genutzte genKI detailliert in einem Anhang in meiner Seminararbeit aufgeführt und die Zitate in der Seminararbeit deutlich gekennzeichnet habe. Dieser Anhang ist Teil meiner Seminararbeit. Ich bin für ggfs. durch genKI generierte Inhalte, die Einhaltung urheberrechtlicher Bestimmungen, meine eigenständige Erstellung sowie für die wissenschaftliche Integrität meiner Seminararbeit selbst verantwortlich. Mir ist bekannt, dass fehlende oder fehlerhafte Angaben als Täuschungsversuch gewertet werden können. Ich erkläre, dass ich die Bestimmungen zum Urheberrecht und Datenschutz (DSGVO) sowie die jeweils geltenden Richtlinie der Fakultät für Wirtschaftsinformatik zur Anwendung von genKI-Tools erfüllt habe und erfüllen werde.

\makeatletter
Hamburg, den {\@date}
\makeatother

\vspace{2cm}
\rule{6cm}{0.25pt}\\
\makeatletter
Valentina Ermisch \par
\makeatother

\vspace{2cm}
\rule{6cm}{0.25pt}\\
\makeatletter
Lisa-Sophie Kaisik \par
\makeatother

\newpage
\thispagestyle{empty}
\null
\newpage

% --------------------------------------------------------------------------------
% ----- Literaturliste (Muster)
% --------------------------------------------------------------------------------
\newpage
\thispagestyle{empty}
\label{sec:literaturliste}
\par\textbf{\textsf{Thema:}} Wirtschaftsinformatik und Nachhaltigkeit: Anwendungsszenarien in der Transportwirtschaft
\par\textbf{\textsf{Bearbeiter:}} Valentina Ermisch, Lisa-Sophie Kaisik
\par\textbf{\textsf{Datum:}} \today
\bigskip
% % ====> Delete me
% \begin{tikzpicture}[overlay]
%     \node[draw, blue, font=\sffamily\Large, xshift=80mm, yshift=-6mm, rounded corners=1mm]{Muster der Literaturliste};
% \end{tikzpicture}
% % <==== /Delete me

\section*{Literaturliste}

% ----- Nachfolgend eine händisch gesetzte Literaturliste, die sich exakt an die Syntax im Abschnitt \ref{sec:literaturhowto} hält. Wir nutzen diese aber hier nicht, sondern lassen BibLaTeX die Einträge formatieren.
\iffalse
David Chaum: Untraceable Electronic Mail, Return Addresses, and Digital Pseudonyms. Communications of the ACM 24/2 (1981) 84--88.

David Chaum: The Dining Cryptographers Problem: Unconditional Sender and Recipient Untraceability. Journal of Cryptology 1/1 (1988) 65--75.

David Goldschlag, Michael Reed, Paul Syverson: Onion Routing for Anonymous and Private Internet Connections. Communications of the ACM 42/2 (1999) 39--41.

Andreas Pfitzmann: Diensteintegrierende Kommunikationsnetze mit teilnehmerüberprüfbarem Datenschutz. IFB 234, Springer-Verlag, Berlin 1990.

Wei Wang, Mehul Motani, Vikram Srinivasan: Dependent link padding algorithms for low latency anonymity systems. Proc. 15th ACM conference on Computer and communications security. ACM, 2008, 323--332.
\fi

% ----- Nachfolgend die Ausgabe unter Verwendung von BibLaTeX. Die Formatierung übernimmt BibLaTeX. Dadurch wird es zu Abweichungen von der vorgegebenen Syntax kommen. Dies ist tolerabel, da es i.W. auf Einheitlichkeit ankommt, nicht auf eine dogmatische Einhaltung der Syntax.
\fullcite{AmmoserHoppe2006}

\fullcite{AnalysisTransportSDGs}

\fullcite{BMZ2025}

\fullcite{BVL2024}

\fullcite{Bundestag2025}

\fullcite{EY2023}

\fullcite{Fareed2024}

\fullcite{FraunhoferIML2014}

\fullcite{Greenpeace2025}

\fullcite{Hausladen2020}

\fullcite{HLAG2016}

\fullcite{LpB-BW2023}

\fullcite{RIOHGIZ2022}

\fullcite{SAP2025}

\fullcite{UBA2024}

\fullcite{UBA2025}

\fullcite{UN2021}

\fullcite{UNCTAD2018}

\fullcite{UNRIC2025}

\fullcite{WEF2025}

\fullcite{SchmalzriedRitzrau2022ITNachhaltigkeit}

\fullcite{KadlubekEtAl2024ITS}

\fullcite{VanBonnLastring2014BigDataSupplyChain}

\fullcite{Deckert2021CSRUndLogistik}

\fullcite{Kayikci2024DigitalizationResilienceCargo}

\fullcite{BretzkeBarkawi2012NachhaltigeLogistik}

\fullcite{SchroeerEtAl2023ABChain}



% --------------------------------------------------------------------------------
% ----- Wiss. Kurzzusammenfassung (Muster)
% --------------------------------------------------------------------------------
%\newpage
%\thispagestyle{empty}
%\label{sec:kurzusammenfassung}
%\par\textbf{\textsf{Thema:}} Privacy Enhancing Technologies zum Schutz von Kommunikationsbeziehungen
%\par\textbf{\textsf{Bearbeiter:}} Eva Musterfrau, Heinz Mustermann
%\par\textbf{\textsf{Datum:}} \today
%\bigskip
% ====> Delete me
%\begin{tikzpicture}[overlay]
	%\node[draw, blue, font=\sffamily\Large, xshift=60mm, yshift=-6mm, rounded corners=1mm]{Muster der Wiss. Kurzzusammenfassung};
	%\node[font=\sffamily\small\itshape, xshift=72mm, yshift=-14mm]{Umfang: 1-3 Seiten, wenn keine konkrete Vorgabe};
%\end{tikzpicture}
% <==== /Delete me
%\section*{Überschrift}

%Lorem ipsum dolor sit amet, consectetur adipisicing elit, sed do eiusmod tempor incididunt ut labore et dolore magna aliqua. Ut enim ad minim veniam, quis nostrud exercitation ullamco laboris nisi ut aliquip ex ea commodo consequat. Duis aute irure dolor in reprehenderit in voluptate velit esse cillum dolore eu fugiat nulla pariatur. Excepteur sint occaecat cupidatat non proident, sunt in culpa qui officia deserunt mollit anim id est laborum.

%Lorem ipsum dolor sit amet, consectetur adipisicing elit, sed do eiusmod tempor incididunt ut labore et dolore magna aliqua. Ut enim ad minim veniam, quis nostrud exercitation ullamco laboris nisi ut aliquip ex ea commodo consequat. Duis aute irure dolor in reprehenderit in voluptate velit esse cillum dolore eu fugiat nulla pariatur. Excepteur sint occaecat cupidatat non proident, sunt in culpa qui officia deserunt mollit anim id est laborum.

%Lorem ipsum dolor sit amet, consectetur adipisicing elit, sed do eiusmod tempor incididunt ut labore et dolore magna aliqua. Ut enim ad minim veniam, quis nostrud exercitation ullamco laboris nisi ut aliquip ex ea commodo consequat. Duis aute irure dolor in reprehenderit in voluptate velit esse cillum dolore eu fugiat nulla pariatur. Excepteur sint occaecat cupidatat non proident, sunt in culpa qui officia deserunt mollit anim id est laborum.

%\section*{Überschrift}

%Lorem ipsum dolor sit amet, consectetur adipisicing elit, sed do eiusmod tempor incididunt ut labore et dolore magna aliqua. Ut enim ad minim veniam, quis nostrud exercitation ullamco laboris nisi ut aliquip ex ea commodo consequat. Duis aute irure dolor in reprehenderit in voluptate velit esse cillum dolore eu fugiat nulla pariatur. Excepteur sint occaecat cupidatat non proident, sunt in culpa qui officia deserunt mollit anim id est laborum.

%Lorem ipsum dolor sit amet, consectetur adipisicing elit, sed do eiusmod tempor incididunt ut labore et dolore magna aliqua. Ut enim ad minim veniam, quis nostrud exercitation ullamco laboris nisi ut aliquip ex ea commodo consequat. Duis aute irure dolor in reprehenderit in voluptate velit esse cillum dolore eu fugiat nulla pariatur. Excepteur sint occaecat cupidatat non proident, sunt in culpa qui officia deserunt mollit anim id est laborum.

%Lorem ipsum dolor sit amet, consectetur adipisicing elit, sed do eiusmod tempor incididunt ut labore et dolore magna aliqua. Ut enim ad minim veniam, quis nostrud exercitation ullamco laboris nisi ut aliquip ex ea commodo consequat. Duis aute irure dolor in reprehenderit in voluptate velit esse cillum dolore eu fugiat nulla pariatur. Excepteur sint occaecat cupidatat non proident, sunt in culpa qui officia deserunt mollit anim id est laborum.

% --------------------------------------------------------------------------------
% ----- Todo list
% --------------------------------------------------------------------------------
%\listoftodos
% \todototoc

\end{document}
