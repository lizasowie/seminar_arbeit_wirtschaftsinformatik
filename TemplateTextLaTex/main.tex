%!TEX encoding = UTF-8 Unicode
% --------------------------------------------------------------------------------
\documentclass[
	fontsize=12pt,
	headings=small,
	parskip=half,           % Ersetzt manuelles Setzen von parskip/parindent.
	bibliography=totoc,
	numbers=noenddot,       % Entfernt den letzten Punkt der Kapitelnummern.
	open=any,               % Kapitel kann auf jeder Seite beginnen.
%	final                   % Entfernt alle todonotes und den Entwurfstempel.
]{scrreprt}
% --------------------------------------------------------------------------------
% Hinweis: Das Übersetzen dieser Datei funktioniert auch Online mit 
% https://www.overleaf.com. Hierzu müssen neben dieser Datei im gleichen 
% Verzeichnis die Dateien hinweiseabschlussarbeit.bib und stylesvs.tex liegen.
% --------------------------------------------------------------------------------
\input{stylesvs}

\title{Wirtschaftsinformatik und Nachhaltigkeit:\\
Anwendungsszenarien in der Transportwirtschaft}
\author{Valentina Ermisch, Lisa-Sophie Kaisik}
% \date{01.01.2015} % Für bestimmtes Datum diese Zeile aktivieren.

\begin{document}

\newpage
\thispagestyle{empty}
\null
\newpage

% --------------------------------------------------------------------------------
% ----- Muster für weitere Deckblätter siehe hinten
% --------------------------------------------------------------------------------
\begin{titlepage}
\mbox{\parbox[t][1.75cm][b]{2.2cm}{\uhhlogo}}
\begin{center}\Large
	\vfill Seminararbeit
	\vfill \makeatletter {\Large\textsf{\textbf{\@title}}\par} \makeatother
	\vfill vorgelegt von \par\bigskip
	\begin{tabu}{p{0.5\textwidth}p{0.5\textwidth}}
	\centering Valentina Ermisch     & \centering Lisa-Sophie Kaisik \\[1ex]
	\centering Matrikelnummer 7775958 & \centering Matrikelnummer 7726396 \\
	\centering Studiengang Betriebs- & \centering Studiengang Wirtschafts- \\
	\centering wirtschaftslehre & \centering informatik \\

	\end{tabu}
	\vfill MIN-Fakultät \par Fachbereich Wirtschaftsinformatik
	\vfill \makeatletter eingereicht am {\@date} \makeatother
	\vfill Betreuer: Prof. Dr. Markus Nüttgens
\end{center}
\ifoptionfinal{}{
\begin{tikzpicture}[remember picture, overlay]
	\node[draw, red, font=\ttfamily\bfseries\Large, xshift=30mm, yshift=238mm, rotate=340, text centered, text width=6cm, very thick, rounded corners=4mm] at (current page.south) {Entwurf vom \today};
\end{tikzpicture}
%\begin{tikzpicture}[overlay]
%	\node[draw, blue, font=\sffamily\Large, xshift=0mm, yshift=242mm, rotate=0, text centered, rounded corners=1mm] at (current page.south) {Muster des Deckblatts für Seminararbeiten};
%\end{tikzpicture}
}
\end{titlepage}

% --------------------------------------------------------------------------------
% ----- Ab hier folgt der Haupttext
% --------------------------------------------------------------------------------

%\chapter*{Aufgabenstellung}

%Soweit eine ausformulierte Aufgabenstellung mit der Betreuerin bzw. dem Betreuer vereinbart wurde, diese bitte hier einfügen.

%\chapter*{Vorwort, Zusammenfassung}

%Für die eilige Leserin bzw. den eiligen Leser sollen auf etwa einer halben, maximal einer Seite die wichtigsten Inhalte, Erkenntnisse, Neuerungen bzw. Ergebnisse der Arbeit beschrieben werden.

%Durch eine solche Zusammenfassung (im Engl. auch Abstract genannt) am Anfang der Arbeit wird die Arbeit deutlich aufgewertet. Hier sollte vermittelt werden, warum man die Arbeit lesen sollte.

\pagenumbering{roman}

\tableofcontents

\cleardoublepage
\pagenumbering{arabic}
\setcounter{page}{1}

\chapter{Einleitung}

\section{Motivation und Problemstellung}

Von Jahr zu Jahr wird es immer wärmer, weshalb es wichtiger denn je wird, den Zielen des Pariser Klimaabkommens näherzukommen. Das Ziel ist es, den Temperaturanstieg unter 2°C zu halten. Dazu ist es besonders wichtig, die CO$_2$-Emissionen zu senken, die zu einem Großteil durch den Transportsektor verursacht werden. Neben den klimatischen Herausforderungen entwickelt sich unsere Gesellschaft immer weiter. Die Kundenanforderungen steigen und die Komplexität der Logistik wächst. Der Transportsektor steht daher vor der Herausforderung, nachhaltige Lösungen entlang logistischer Prozesse zu entwickeln. Dabei soll nicht nur effizient, sondern auch nachhaltig weiterentwickelt werden. 
Ein zentraler Aspekt zur Gestaltung einer effizienten und nachhaltigen Transportwirtschaft können verschiedene Lösungsansätze wie optimierte Routen durch KI, maschinelles Lernen oder intelligente Transportsysteme sein. Das schafft ökologische sowie ökonomische Vorteile.

\section{Zielsetzung}

Das Ziel dieser Seminararbeit ist es, zu erklären, wie IT-gestützte Logistik mehr Potenzial zur Realisierung von Effizienz, Kosteneffektivität und Nachhaltigkeit beiträgt, um den steigenden Anforderungen gerecht zu werden. Es sollen mögliche Anwendungen von Transportmanagementsystemen, intelligenten Transportsystemen und Lösungen durch künstliche Intelligenz dargestellt werden. Dabei soll ihr Nutzen für die Weiterentwicklung der Transportwirtschaft deutlich werden. Im Folgenden nehmen wir nur Bezug auf drei der 17 Sustainable Development Goals (SDGs), da sich unsere Arbeit primär mit nachhaltiger Energie, Infrastruktur und der Reduktion der CO$_2$-Emissionen im Transportsektor befasst. 
\section{Aufbau der Semianrarbeit}

Zunächst werden die Grundlagen der Bereiche Transportwirtschaft, Nachhaltigkeit und IT-Technologien der Wirtschaftsinformatik erläutert. Es werden Herausforderungen der Transportwirtschaft aufgeführt, die aus unserer stetig wachsenden Gesellschaft resultieren. Der Bereich Nachhaltigkeit zeigt die Anforderungen an zukünftige Logistik, führt Ziele auf und unterscheidet zwischen grüner Logistik und nachhaltiger Logistik. Die Grundlagen der Wirtschaftsinformatik definieren erste Begrifflichkeiten für ein Grundverständnis des Hauptteils. Es geht um die Begriffe, E-Logistik, Informations- und Kommunikationstechnologien (IKT), Smart Logistics, Logistik 4.0, automatisierte und intelligente Verwaltung, sowie Technologien wie Internet of Things (IoT), Big Data, Cloud Computing, künstliche Intelligenz (KI) und Blockchain.

Im darauf folgenden Hauptteil werden die einzelnen Möglichkeiten zur Steigerung von Effizienz und Nachhaltigkeit in der Transportwirtschaft genauer aufgeführt. Es geht um strukturierte Routenoptimierung durch Transportation Management-Systeme (TMS). Es wird der Einsatz von intelligenten Transportsystemen beschrieben, welche ihre Daten aus einem System namens Internet of Things (IoT) ziehen. Zuletzt geht es um den Einsatz von Blockchain-Technologien und Smart Contracts in Lieferketten. 

Anhand eines konkreten Anwendungsbeispiels wird die Umsetzung der aufgezeigten Technologien durch das Unternehmen Amazon Web Services Inc. erläutert und bewertet. Es wird nach den drei Säulen der Nachhaltigkeit und einer Abwägung zwischen Problemlösungen und neu entstehenden Problemen abgewogen.

Zuletzt folgt eine Zusammenfassung der Kernergebnisse und eine Auswertung, ob die genannten IT-Lösungen einen Beitrag zur effizienten und nachhaltigen Entwicklung des Transportsektors leisten. Im Ausblick werden zukünftige Fragen und neue Herausforderungen, die sich durch die Nutzung von IT-Lösungen ergeben, aufgeführt.




\chapter{Grundlagen}


\section{Transportwirtschaft und ihre Herausforderungen}

\todo{Definition Logistik: System zur optimalen Versorgung mit Materialien}
\todo{duale Strukturiertheit der Logistik, raum-Zeit-Güter und die Bedeutung des Informationsmanagements }
\todo{8 Rs der Logsitik, grundlegendes Ziel}
\todo{Einordnung und Abgrenzung zum Supply Chain Managemenr (SCM), wobei die Logistik Teildisziplin ist}

\section{Nachhaltigkeit mit Bezug auf Transportwirtschaft}


Der Begriff der Nachhaltigkeit besteht aus drei Hauptkomponenten: aus der ökologisch, der ökonomisch und der sozialen. Die ökologische Nachhaltigkeit beinhaltet den Schutz des Klima, der Biodiversität und die bedachte Nutzung der Ressourcen. Die ökonomische Komponente zeichnet den bedachten Konsum von Gütern aus. Die soziale Nachhaltigkeit setzt auf  Bildung, Gesundheit und Chancengleichheit für jeden. \cite{Greenpeace2025}

Agenda 2030 setzt sich mit der Nachhaltigkeit auseinander. Dabei umfasst diese 17 Ziele, auch Sustainable Development Goals oder SDGs genannt. Diese traten am 01.01.2016 in Kraft und erweiterten somit die Ziele der vorherigen Agenda 21. 193 UN-Mitgliedstaaten haben diese verabschiedet, jedoch ist die Agenda 2030 nur eine (freiwillige) Selbstverpflichtung für die jeweiligen Länder. \cite{LpB-BW2023}

SDG 7, 12 und 13 sind die drei Ziele, die einen direkten Bezug zum Klimaschutz haben. Dabei schreibt das SDG 7 “Bezahlbare und saubere Energie”, dass bis 2030 jeder Zugang zu nachhaltiger und moderner Energie haben sollte. Darüber hinaus muss die Energie bezahlbarer und verlässlicher werden. \cite{LpB-BW2023}
Das SDG 12 “Nachhaltige/r Konsum und Produktion” setzt voraus, dass der Mensch einem verantwortungsvollen Konsum nachgeht und nicht im Überfluss lebt. Dabei wird auf eine funktionale Kreislaufwirtschaft gesetzt. (17 Ziele)
Der Klimaschutz wird mit dem SDG 13 "Maßnahmen zum Klimaschutz” abgedeckt. Dieses schreibt vor, dass bis 2030 die Treibhausgasemissionen um 42 Prozent gesenkt werden müssen, damit die globale Erwärmung nicht über 1,5 Grad Celsius steigt. (17 Ziele)


\todo{Tripple Bottom Line, drei Säulen}
\todo{Definition und Untrscheidung Green Logistic (Fokus auf nur Ökologie, CO2) und Nachhaltiger Logistik (alle drei Säulen)}
\todo{Notwendigkeit der Quantifizierung der Auswirkunge, ISO14083 Berechnung von Treibhausgas Emissionen THG in Transportketten}
\todo{Einsatz von IKT im Kontext von Reduzierung der Treibhausgase}

\newpage
\section{Wirtschaftsinformatik und Smart Logistics}

\todo{Definition (E-Logsitik): Planung, Steuerung Überwachung der Flüsse als Business Lösung / System}
\todo{IT IKT erfolgreichste Möglichkeit mit wieterbringenden Lösungen}
\todo{Definition Smart Logistics für Logistik 4.0}
\todo{Optimierung, Erreichung von Nachhaltigekti durch automatisierte und intelligente Verwaltung logischer Operationen}
\todo{Alle möglichen technologischen Treiber einführen: IoT, Big Data, Cloud Computing, KI}

\chapter{Anwendungsszenarien für IT-gestützte Lösungsansätze}

\section{Transportation Management Systeme und Enterprise Resource Planning}

In der modernen Transportwirtschaft bilden Enterprise-Resource-Planning (ERP)- und Transport-Management-Systeme (TMS) die Grundlage für die Koordination physischer Warenströme und begleitender Datenflüsse. \cite{Hausladen2020} Für Geschäftsführer sind diese Systeme nicht nur operative Werkzeuge, sondern bieten die Möglichkeit der „Twin Transformation“. Sie verknüpfen ökonomische Effizienzsteigerung direkt mit ökologischen Nachhaltigkeitszielen. [Ern23] Durch die Integration lässt sich die Komplexität globaler Netzwerke reduzieren, was die Basis für eine nachhaltige Standortsicherung und Wettbewerbsfähigkeit bildet. [Kad+24]

Die Architektur moderner ERP-Systeme ist modular aufgebaut und zielt auf die unternehmensweite Integration aller Ressourcen wie Personal, Finanzen und Material ab. [Hau20] Das TMS übernimmt innerhalb dieser Struktur die Planung, Durchführung und Überwachung von Transportprozessen. [BL14]
Die Integration von Auftrags- und Fuhrparkmanagement schafft durch die Verknüpfung von Auftragsdaten aus dem ERP mit den operativen Kapazitäten des TMS eine medienbruchfreie Prozesskette von der Bestellung bis zur Auslieferung. [Hau20] Das zentrale Datenmanagement dient als Single Source of Truth, konsolidiert heterogene Datenbestände, vermeidet Redundanzen und sichert hohe Datenqualität für die Entscheidungsunterstützung. [Hau20] Moderne Architekturen nutzen zudem Cloud-Lösungen und standardisierte Schnittstellen (EDI/XML), um externe Partner und mobile Endgeräte in Echtzeit zu integrieren.

Algorithmische Verfahren in TMS-Lösungen sind elementar, um ökonomische und ökologische Ziele zu verbinden. [SR22] Da der Transport rund 90\% der logistikbedingten Treibhausgasemissionen verursacht, führt jede Routenoptimierung zu einer messbaren Umweltentlastung. [Dec21]
Mathematische Algorithmen zur Ressourcenproduktivität erhöhen die Auslastung von Transportmitteln und reduzieren gefahrene Kilometer. Praxisbeispiele wie bei Bosch zeigen Einsparungen von bis zu 15\% der Logistikkosten. [SR22] Die Senkung des Kraftstoffverbrauchs erfolgt durch optimierte Tourenplanung und die Vermeidung von Leerlauf. Schulungen und intelligente Systeme können den Dieselverbrauch um bis zu 5 Liter pro 100 km senken. [Dec21] IT-gestützte Planung beim Modal Split ermöglicht die Kombination verschiedener Verkehrsträger (Schiene, Straße, Wasser) und verringert so die Umweltauswirkungen insgesamt. [Dec21]
Der Einsatz von Künstlicher Intelligenz (KI) und Advanced Analytics hebt die Netzwerkoptimierung auf eine vorausschauende Ebene. [RD22] Während klassische Systeme reaktiv planen, ermöglichen datenbasierte Prognosen proaktives Handeln. [BL14]
KI-gestützte Analysen historischer Daten in der prognosebasierten Kapazitätsplanung helfen, Nachfrageschwankungen früh zu erkennen und Kapazitäten gezielt bereitzustellen. [ey-studie-digital-und-nachhaltig-die-zukunft-sichern-februar-2023.pdf] Intelligente Logistikplattformen zur Vermeidung von Leerfahrten bündeln Fracht- und Laderaumangebote branchenübergreifend und senken Leerfahrten im Fernverkehr auf unter 10\%. [Dec21] Die Echtzeit-Entscheidungsunterstützung durch die Verarbeitung von Sensordaten (IoT) ermöglicht automatische Routenanpassungen bei Störungen, was die Resilienz der Lieferketten stärkt und unnötige Umwege vermeidet. [Kc24]


\section{Intelligente Transportsysteme und Internet of Things}

Die Logistik ist zunehmend auf Informations- und Kommunikationstechnologien (IKT) angewiesen, um wachsende Warenströme und die dazugehörigen Informationsflüsse effizient zu steuern. [Hau20] Intelligente Transportsysteme (ITS) stellen dabei eine zentrale Verbindung zwischen Informatik und Telekommunikation her und erhöhen die Leistungsfähigkeit, Sicherheit und Wirtschaftlichkeit von Transportprozessen. [AH06, Hau20] Für Geschäftsführer eines Transportunternehmens können Investitionen in Internet-of-Things (IoT)-Lösungen nicht nur Kostenvorteile bringen, sondern auch neue strategische Entwicklungsmöglichkeiten eröffnen.

IoT beschreibt ein Netzwerk physischer Objekte mit Sensoren, Software und Internetanbindung, das Daten erfasst, austauscht und mit anderen Systemen kommuniziert. Es trägt zur Ressourcenproduktivität bei, indem es Transparenz schafft und eine Echtzeitsteuerung ermöglicht. Dadurch entstehen erhebliche Effizienzgewinne bei gleichzeitig geringeren ökologischen Fußabdruck des Unternehmens. [SR22]

Durch die Vernetzung von Fahrzeugen über IoT-Sensoren und GPS können Disponenten Transportwege dynamisch an die aktuelle Verkehrslage anpassen. [Hau20, Kc24] Wirtschaftsinformatische Algorithmen vermeiden Staus und Leerfahrten und senken so Fahrzeit und Energieverbrauch um bis zu 40–70\%. [Kad+24] Ein Praxisbeispiel ist die unternehmensübergreifende Netzwerkoptimierung großer Konzerne, die durch algorithmische Planung bis zu 15\% der Logistikkosten bei deutlich geringerem CO$_2$-Ausstoß reduziert. [BB12, SR22]

IoT-basierte Container nutzen Sensoren und Telematik, um Zustandsdaten wie Temperatur, Feuchtigkeit und Erschütterungen in Echtzeit zu überwachen. [RD22] Besonders in der Kühlkettenlogistik – etwa in der Lebensmittel- oder Pharmabranche – reduziert dies Warenverluste durch Verderb und steigert die Energieeffizienz von Kühlanlagen. [Hau20] Gleichzeitig wird die lückenlose Rückverfolgung zum Standard: Tracking und Tracing sichern die Produktqualität und helfen, regulatorische Anforderungen zuverlässig zu erfüllen. [SJR23]

ITS-Anwendungen im Fahrzeugmanagement unterstützen eine energieeffiziente Fahrweise (Eco Driving) durch die Analyse von Motordaten und die Anzeige optimaler Geschwindigkeiten. [IKad+24] In der Praxis senken digitale Fahrerschulungen den Kraftstoffverbrauch langfristig um bis zu 5 Liter Diesel pro 100 km. [Dec21] Solche Systeme reduzieren Schadstoff- und Lärmemissionen direkt an der Quelle und leisten damit einen messbaren Beitrag zur Umweltleistung. [SR22]
Zukunftsorientierte Konzepte wie das Truck Platooning vernetzen mehrere Lkw zu einem digitalen Konvoi. [2375Mobilizing Sustainable Transport.pdf] Die Fahrzeuge halten automatisch den optimalen Abstand, wodurch sich der Luftwiderstand verringert und Kraftstoff eingespart wird. Das trägt direkt zur Dekarbonisierung des Straßengüterverkehrs bei. [Transportation Report 2021FullReportDigital.pdf] Pilotprojekte, etwa in Singapur, zeigen, dass diese Systeme zugleich dem Fachkräftemangel entgegenwirken und die Verkehrssicherheit erhöhen, da menschliche Fehler reduziert werden.

IoT und ITS ermöglichen den Übergang zu einer intelligenten, datenbasierten Logistik, in der nahezu alle Prozesse in Echtzeit steuerbar sind. Dadurch entsteht die Grundlage für eine nachhaltige Transformation und langfristige Wettbewerbsfähigkeit von Transportunternehmen. [RD22, SR22]

\section{Plattformökonomie und datengetriebene Kooperationsmodelle}

Neben TMS und ERP ermöglichen auch Plattformmodelle die Umsetzung der “Twin-Transformation” in der von Volatilität und hohen Kostendruck geprägten Logistikwelt. [Transportation Report 2021FullReportDigital.pdf, Ern23] Für Geschäftsführer stellen sie ein zentrales Steuerungsinstrument dar, um die Fragmentierung des Transportsektors zu überwinden und Managemententscheidungen datengestützt und proaktiv zu treffen. [Transportation Report 2021FullReportDigital.pdf] Digitale Plattformen transformieren die Logistik von einer Dienstleistung zu einem wertschöpfenden, integrierten Netzwerkmanagement. [Hau20]
Digitale Frachtbörsen fungieren als virtuelle Marktplätze, die Angebot und Nachfrage für Transportkapazitäten in Echtzeit zusammenführen. [Hau20] Spotmärkte decken kurzfristige Bedarfe ab, während Kontaktmärkte langfristige Kooperationen ermöglichen. Beide Ansätze verbessern die Ressourcennutzung und lösen starre zweiseitige Verträge ab. [Hau20]
Durch intelligente Algorithmen werden Ladungen unternehmensübergreifend gebündelt, was im Straßengüterverkehr die Markttransparenz erhöht und den Zugang zu verfügbaren Kapazitäten erleichtert. [Kad+24, Dec21] 
Praxisbeispiele wie Teleroute oder TRANS.eu zeigen, dass Transportunternehmen so ihre operative Flexibilität steigern und Laderaum europaweit effizienter nutzen können. [Hau20]

Die Leistungsfähigkeit moderner Kooperationsmodelle hängt von der Integration unterschiedlicher Systeme wie ERP, TMS und Telematik ab. Cloud-Plattformen wie GT Nexus oder AX4 schaffen eine Single Source of Truth, indem sie Statusmeldungen und Dokumente über Unternehmensgrenzen hinweg in einer einheitlichen Datenumgebung zusammenführen. [Hau20]
Einheitliche Datenstandards wie EDI oder XML sind dabei entscheidend, um Medienbrüche zu vermeiden und eine vollständige Transparenz der Warenströme sicherzustellen. Diese Plattformen wirken wie ein digitales Nervensystem, das den physischen Transportfluss nahtlos mit dem Informationsfluss verbindet. [1-s2.0-S0268401224000021-main.pdf, Hau20]

Plattformgestützte Kooperationen erhöhen die Ressourcenproduktivität messbar. Durch die Reduktion von Leerfahrten im Fernverkehr auf unter 10\% können hohe Effizienzgewinne erzielt werden. [Dec21] Unternehmen wie Bosch zeigen, dass sich durch algorithmische Netzwerkplanung bis zu 15\% der Logistikkosten einsparen lassen. Gleichzeitig sinken Energieverbrauch und Treibhausgasemissionen um 30 bis 50\%. [Kad+24, SR22]
Die durch Plattformen gewonnene Transparenz ermöglicht eine vorausschauende („Predictive“) Logistik. Auf Basis verifizierter Daten lassen sich Fahrten präziser planen und Transportwege verkürzen. Das senkt den ökologischen Fußabdruck und stärkt zugleich die Wettbewerbsfähigkeit von Transportunternehmen. [UNC18, SR22]


\section{Blockchain Technologien}

Logistik und Supply Chain Management (SCM) hängen heutzutage stark von der Fähigkeit ab, fälschungssichere Informationen über die gesamte Lieferkette hinweg auszutauschen. [SJR23] Für Geschäftsführer von Transportunternehmen rückt das Thema Transparenz in den Fokus, da Kunden und Stakeholder zunehmend verifizierte Nachweise über Produktionsbedingungen, Herkunft und ökologische Auswirkungen von Produkten fordern. Die Blockchain dient dabei als technologische Grundlage, um von einer reaktiven Informationsweitergabe zu einer proaktiven Steuerung dynamischer Wertschöpfungsketten überzugehen. [SJR23] Man kann sich die Blockchain in einer Lieferkette wie ein gemeinsames Kassenbuch vorstellen, in dem jede Partei ihre Einträge mit unlöschbarer Tinte vornimmt. Da alle Teilnehmer identische Kopien besitzen, erkennt das System sofort, wenn jemand eine Seite manipulieren will. So entsteht Vertrauen in ein globales Netzwerk, ohne dass eine zentrale Instanz nötig ist.

Die Blockchain-Technologie basiert auf einer dezentralen Datenhaltung, bei der alle Teilnehmer des Netzwerks eine Kopie des Transaktionsverzeichnisses speichern. [Hau20, SJR23] Dies eliminiert den Bedarf an zentralen Intermediären und erhöht die Datensicherheit, da Informationen bei Systemausfällen an anderen Knotenpunkten bestehen bleiben. [SJR23] Die Unveränderbarkeit der Daten wird durch kryptografische Hashfunktionen sichergestellt. Sobald ein Datenblock erstellt ist, kann er nicht mehr gelöscht oder verändert werden, ohne dass das gesamte Netzwerk die Manipulation erkennt. [SJR23]
Ein zentraler betriebswirtschaftlicher Hebel liegt in Smart Contracts. Dabei handelt es sich um automatisierte Wenn‑Dann‑Beziehungen im Code, die Geschäftsprozesse wie Zahlungen bei Wareneingang oder Zollmeldungen selbstständig ausführen können – ohne manuellen Eingriff. [Hau20, SJR23]

Blockchain‑Anwendungen ermöglichen eine lückenlose Produktrückverfolgung über den gesamten Lebenszyklus hinweg. [SJR23] Im Kontext der Nachhaltigkeit erlaubt die Technologie eine manipulationssichere Erfassung und Verteilung von Primärdaten zu CO$_2$‑Emissionen entlang der gesamten Transportkette. [SJR23] Durch automatisierte Verifizierungen („Verifiable Proofs“) wird der administrative Aufwand für die CO$_2$-Bilanzierung deutlich reduziert, während die Glaubwürdigkeit gegenüber Kunden steigt. [SJR23]


Für kleine und mittlere Unternehmen entstehen durch mehr Datensouveränität und den Verzicht auf Intermediäre deutliche Kostenvorteile. [SJR23]
Die Möglichkeit, Nachhaltigkeitsnachweise – etwa für das Lieferkettensorgfaltspflichtengesetz – rechtskonform bereitzustellen, schafft einen strategischen Wettbewerbsvorteil. [SJR23]
Zu den Barrieren zählen ein hoher Initialaufwand und eine geringe Datenfreigabebereitschaft der Partner, die Wettbewerbsbedenken haben. [SJR23] Zudem erfordert die dezentrale Speicherung großer Datenmengen erhebliche Serverkapazitäten, was zu höherem Energieverbrauch führen kann. [SJR23]
Technologische Risiken wie Cyberangriffe auf Schnittstellen müssen über ein integriertes IT‑Sicherheitsmanagement gesteuert werden. [SJR23]

Beispielhaft haben Maersk und IBM eine offene Digitalplattform entwickelt, die grenzüberschreitende Warenflüsse transparent abbildet und papierlose Prozesse ermöglicht. Hyundai Merchant Marine testete ein Blockchain‑basiertes System zur Optimierung der Containerlogistik und zur Überwachung von Lieferketten in Echtzeit.
Projekte wie SiLKe nutzen die Blockchain zur Rückverfolgung von Lebensmitteln, um im Krisenfall – etwa bei Rückrufen – schneller und gezielter reagieren zu können. [SJR23]

\section{Predictive Logistics und Künstliche Intelligenz}

In einem volatilen Marktumfeld entscheidet die Leistungsfähigkeit der Logistik über die Stabilität globaler Lieferketten. Versagt sie, drohen Unterbrechungen in der gesamten Wertschöpfungskette. [1-s2.0-S2949899624000042-main.pdf] Für Geschäftsführer ist es wichtig, den Schritt von reaktiver Hektik zu vorausschauender Steuerung zu vollziehen. [BL14, Hau20]
Durch die Verknüpfung interner ERP-Daten mit externen Informationsquellen lassen sich lokale Effizienzgrenzen überwinden. Auf diese Weise können bis zu 75\% der strukturellen Verbesserungspotenziale im Netzwerk genutzt werden. [BL14] Man kann sich Predictive Logistics wie einen erfahrenen Schachgroßmeister vorstellen. Während ein traditionelles System nur auf den aktuellen Zug reagiert, berechnet die KI hunderte Züge im Voraus. Sie erkennt drohende Hindernisse, wägt Optionen ab und wählt den Weg, der Ressourcen spart und das Ziel mit minimalem Risiko erreicht.

Predictive Logistics nutzt Algorithmen, um auf Basis historischer Transaktionen und aktueller Echtzeitdaten das Verhalten von Kunden, Fahrern und Märkten vorherzusagen. [UNC18]
Im Straßengüterverkehr ermöglichen IoT-Sensoren und Telematiksysteme die ständige Überwachung von Fahrzeugzuständen wie Standort oder Temperatur. Externe Daten wie Wetterprognosen oder Verkehrsanalysen fließen in die Modelle ein, damit das System Störungen vorausschauend erkennt. [UNC18]
Diese vorausschauende Planung erlaubt eine präzisere Bestandsführung und verbessert die Planbarkeit der Warenströme erheblich. [BL14, Hau20]

Künstliche Intelligenz (KI), insbesondere Maschinelles Lernen (ML), identifiziert in großen Datenmengen Muster und Zusammenhänge und entwickelt darauf aufbauend autonome Handlungsstrategien. [Hau20] Neuronale Netze und Deep‑Learning‑Modelle erkennen komplexe Nachfragemuster oder Fahrverhalten und bilden so einen digitalen Schatten der Logistikprozesse, der Echtzeit-Entscheidungen ermöglicht. [Hau20] In TMS-Umgebungen berechnen selbstlernende genetische Algorithmen unter Berücksichtigung von Zeit, Kosten und Emissionen die effizienteste Route. [Role and Applications of Advanced Digital Technologies in Achieving Sustainability in Multimodal Logistics Operations.pdf] KI-gestützte Systeme führen Aufgaben wie Routenplanung oder Bestellprognosen zunehmend autonom aus. [Developing Smart Logistics for.pdf]

Der Einsatz von KI ist ein wesentlicher Hebel zur Steigerung der Ressourcenproduktivität. [SR22] KI-Analysen ermöglichen die Vermeidung von Leerfahrten durch prädiktive Fahrzeugdisposition und gemeinsame Auslastung über Unternehmensgrenzen hinweg. Dadurch können Leerfahrten im Fernverkehr auf unter 10\% sinken. [UNC18, Dec21] Intelligente Verkehrssteuerung und Fahrerassistenzsysteme reduzieren Wartezeiten, Kraftstoffverbrauch und Emissionen um 30 bis 50\%. [Kad+24f, Kc24] Vorausschauende Wartung (Predictive Maintenance) nutzt Sensordaten, um Ausfälle früh zu erkennen. Dadurch werden Stillstände vermieden und die Nutzungsdauer von Fahrzeugen und Anlagen verlängert. [Ern23]

KI bietet vor allem kleinen und mittelgroßen Transportunternehmen die Chance, Kosten durch Automatisierung zu senken und die Servicequalität durch präzise Lieferzeitprognosen zu verbessern. [Kc24, SJR23]

Zu den Herausforderungen zählen mangelnde Datenqualität und hohe Anfangsinvestitionen für IT‑Infrastruktur und Fachpersonal. [1-s2.0-S0268401224000021-main.pdf, Kc24] Ein Rebound‑Effekt kann auftreten, wenn Effizienzgewinne durch steigende Transportnachfrage kompensiert werden. [Ern23, SR22] Zudem ist die Akzeptanz im Management oft gering, weil der Return on Investment (ROI) schwer messbar ist. [1-s2.0-S2949899624000042-main.pdf, Kc24]


\chapter{Fazit}

\todo{Zusammenfassung Kernergebnisse, Beantwortung Fragestellung}
\todo{IT gestützte Logistik Sicherung einer nachhaltigen Unternehmensentwicklung}
\todo{Zusammenfassend Bestätigen, dass Logistik und IT Implementierung zusmamengehört}


\chapter{Ausblick}

\cite{FraunhoferIML2014}

\todo{ungelöste Probleme, zukünftige Fragen, weitere quantifizierung der IT Lösungen}
\todo{IT Sicherheit! Cyber Security, Risiken}
\todo{Green IT: Bringt das wirklich Energieeinsparungen oder werden diese nut zur IT weitergetragen}


% --------------------------------------------------------------------------------
% ----- Literaturverzeichnis
% --------------------------------------------------------------------------------
\begin{raggedright} % raggedright schaltet den Blocksatz ab und erzeugt ein stimmigeres Schriftbild im Literaturverzeichnis.
  \printbibliography % alphabetic ist definiert unter biblatex in style.svs
  \label{sec:literaturverzeichnis}
\end{raggedright}

% --------------------------------------------------------------------------------
% ----- Anhang
% --------------------------------------------------------------------------------
\appendix
\setcounter{figure}{0}
\renewcommand\thetable{A.\arabic{figure}}
\setcounter{table}{0}
\renewcommand\thetable{A.\arabic{table}}

% --------------------------------------------------------------------------------
% ----- Eidesstattliche Versicherung
% --------------------------------------------------------------------------------
\chapter*{Eidesstattliche Versicherung}
\vspace{1cm}

Ich erkläre eidesstattlich, dass ich die Arbeit selbständig angefertigt, keine anderen als die angegebenen Hilfsmittel benutzt und alle aus ungedruckten Quellen, gedruckter Literatur oder aus dem Internet im Wortlaut oder im wesentlichen Inhalt übernommenen Formulierungen und Konzepte gemäß den Richtlinien wissenschaftlicher Arbeiten zitiert, durch Fußnoten gekennzeichnet bzw. mit genauer Quellenangabe kenntlich gemacht habe.

Ich versichere, dass auch im Anwendungsfall von generativer Künstlicher Intelligenz (genKI) meine eigene schöpferische Leistung der erhebliche Anteil in dieser Seminararbeit ist und ich die genutzte genKI detailliert in einem Anhang in meiner Seminararbeit aufgeführt und die Zitate in der Seminararbeit deutlich gekennzeichnet habe. Dieser Anhang ist Teil meiner Seminararbeit. Ich bin für ggfs. durch genKI generierte Inhalte, die Einhaltung urheberrechtlicher Bestimmungen, meine eigenständige Erstellung sowie für die wissenschaftliche Integrität meiner Seminararbeit selbst verantwortlich. Mir ist bekannt, dass fehlende oder fehlerhafte Angaben als Täuschungsversuch gewertet werden können. Ich erkläre, dass ich die Bestimmungen zum Urheberrecht und Datenschutz (DSGVO) sowie die jeweils geltenden Richtlinie der Fakultät für Wirtschaftsinformatik zur Anwendung von genKI-Tools erfüllt habe und erfüllen werde.

\makeatletter
Hamburg, den {\@date}
\makeatother

\vspace{2cm}
\rule{6cm}{0.25pt}\\
\makeatletter
Valentina Ermisch \par
\makeatother

\vspace{2cm}
\rule{6cm}{0.25pt}\\
\makeatletter
Lisa-Sophie Kaisik \par
\makeatother

\newpage
\thispagestyle{empty}
\null
\newpage

% --------------------------------------------------------------------------------
% ----- Literaturliste (Muster)
% --------------------------------------------------------------------------------
\newpage
\thispagestyle{empty}
\label{sec:literaturliste}
\par\textbf{\textsf{Thema:}} Wirtschaftsinformatik und Nachhaltigkeit: Anwendungsszenarien in der Transportwirtschaft
\par\textbf{\textsf{Bearbeiter:}} Valentina Ermisch, Lisa-Sophie Kaisik
\par\textbf{\textsf{Datum:}} \today
\bigskip
% % ====> Delete me
% \begin{tikzpicture}[overlay]
%     \node[draw, blue, font=\sffamily\Large, xshift=80mm, yshift=-6mm, rounded corners=1mm]{Muster der Literaturliste};
% \end{tikzpicture}
% % <==== /Delete me

\section*{Literaturliste}

% ----- Nachfolgend eine händisch gesetzte Literaturliste, die sich exakt an die Syntax im Abschnitt \ref{sec:literaturhowto} hält. Wir nutzen diese aber hier nicht, sondern lassen BibLaTeX die Einträge formatieren.
\iffalse
David Chaum: Untraceable Electronic Mail, Return Addresses, and Digital Pseudonyms. Communications of the ACM 24/2 (1981) 84--88.

David Chaum: The Dining Cryptographers Problem: Unconditional Sender and Recipient Untraceability. Journal of Cryptology 1/1 (1988) 65--75.

David Goldschlag, Michael Reed, Paul Syverson: Onion Routing for Anonymous and Private Internet Connections. Communications of the ACM 42/2 (1999) 39--41.

Andreas Pfitzmann: Diensteintegrierende Kommunikationsnetze mit teilnehmerüberprüfbarem Datenschutz. IFB 234, Springer-Verlag, Berlin 1990.

Wei Wang, Mehul Motani, Vikram Srinivasan: Dependent link padding algorithms for low latency anonymity systems. Proc. 15th ACM conference on Computer and communications security. ACM, 2008, 323--332.
\fi

% ----- Nachfolgend die Ausgabe unter Verwendung von BibLaTeX. Die Formatierung übernimmt BibLaTeX. Dadurch wird es zu Abweichungen von der vorgegebenen Syntax kommen. Dies ist tolerabel, da es i.W. auf Einheitlichkeit ankommt, nicht auf eine dogmatische Einhaltung der Syntax.
\fullcite{AmmoserHoppe2006}

\fullcite{AnalysisTransportSDGs}

\fullcite{BMZ2025}

\fullcite{BVL2024}

\fullcite{Bundestag2025}

\fullcite{EY2023}

\fullcite{Fareed2024}

\fullcite{FraunhoferIML2014}

\fullcite{Greenpeace2025}

\fullcite{Hausladen2020}

\fullcite{HLAG2016}

\fullcite{LpB-BW2023}

\fullcite{RIOHGIZ2022}

\fullcite{SAP2025}

\fullcite{UBA2024}

\fullcite{UBA2025}

\fullcite{UN2021}

\fullcite{UNCTAD2018}

\fullcite{UNRIC2025}

\fullcite{WEF2025}

\fullcite{SchmalzriedRitzrau2022ITNachhaltigkeit}

\fullcite{KadlubekEtAl2024ITS}

\fullcite{VanBonnLastring2014BigDataSupplyChain}

\fullcite{Deckert2021CSRUndLogistik}

\fullcite{Kayikci2024DigitalizationResilienceCargo}

\fullcite{BretzkeBarkawi2012NachhaltigeLogistik}

\fullcite{SchroeerEtAl2023ABChain}



% --------------------------------------------------------------------------------
% ----- Wiss. Kurzzusammenfassung (Muster)
% --------------------------------------------------------------------------------
%\newpage
%\thispagestyle{empty}
%\label{sec:kurzusammenfassung}
%\par\textbf{\textsf{Thema:}} Privacy Enhancing Technologies zum Schutz von Kommunikationsbeziehungen
%\par\textbf{\textsf{Bearbeiter:}} Eva Musterfrau, Heinz Mustermann
%\par\textbf{\textsf{Datum:}} \today
%\bigskip
% ====> Delete me
%\begin{tikzpicture}[overlay]
	%\node[draw, blue, font=\sffamily\Large, xshift=60mm, yshift=-6mm, rounded corners=1mm]{Muster der Wiss. Kurzzusammenfassung};
	%\node[font=\sffamily\small\itshape, xshift=72mm, yshift=-14mm]{Umfang: 1-3 Seiten, wenn keine konkrete Vorgabe};
%\end{tikzpicture}
% <==== /Delete me
%\section*{Überschrift}

%Lorem ipsum dolor sit amet, consectetur adipisicing elit, sed do eiusmod tempor incididunt ut labore et dolore magna aliqua. Ut enim ad minim veniam, quis nostrud exercitation ullamco laboris nisi ut aliquip ex ea commodo consequat. Duis aute irure dolor in reprehenderit in voluptate velit esse cillum dolore eu fugiat nulla pariatur. Excepteur sint occaecat cupidatat non proident, sunt in culpa qui officia deserunt mollit anim id est laborum.

%Lorem ipsum dolor sit amet, consectetur adipisicing elit, sed do eiusmod tempor incididunt ut labore et dolore magna aliqua. Ut enim ad minim veniam, quis nostrud exercitation ullamco laboris nisi ut aliquip ex ea commodo consequat. Duis aute irure dolor in reprehenderit in voluptate velit esse cillum dolore eu fugiat nulla pariatur. Excepteur sint occaecat cupidatat non proident, sunt in culpa qui officia deserunt mollit anim id est laborum.

%Lorem ipsum dolor sit amet, consectetur adipisicing elit, sed do eiusmod tempor incididunt ut labore et dolore magna aliqua. Ut enim ad minim veniam, quis nostrud exercitation ullamco laboris nisi ut aliquip ex ea commodo consequat. Duis aute irure dolor in reprehenderit in voluptate velit esse cillum dolore eu fugiat nulla pariatur. Excepteur sint occaecat cupidatat non proident, sunt in culpa qui officia deserunt mollit anim id est laborum.

%\section*{Überschrift}

%Lorem ipsum dolor sit amet, consectetur adipisicing elit, sed do eiusmod tempor incididunt ut labore et dolore magna aliqua. Ut enim ad minim veniam, quis nostrud exercitation ullamco laboris nisi ut aliquip ex ea commodo consequat. Duis aute irure dolor in reprehenderit in voluptate velit esse cillum dolore eu fugiat nulla pariatur. Excepteur sint occaecat cupidatat non proident, sunt in culpa qui officia deserunt mollit anim id est laborum.

%Lorem ipsum dolor sit amet, consectetur adipisicing elit, sed do eiusmod tempor incididunt ut labore et dolore magna aliqua. Ut enim ad minim veniam, quis nostrud exercitation ullamco laboris nisi ut aliquip ex ea commodo consequat. Duis aute irure dolor in reprehenderit in voluptate velit esse cillum dolore eu fugiat nulla pariatur. Excepteur sint occaecat cupidatat non proident, sunt in culpa qui officia deserunt mollit anim id est laborum.

%Lorem ipsum dolor sit amet, consectetur adipisicing elit, sed do eiusmod tempor incididunt ut labore et dolore magna aliqua. Ut enim ad minim veniam, quis nostrud exercitation ullamco laboris nisi ut aliquip ex ea commodo consequat. Duis aute irure dolor in reprehenderit in voluptate velit esse cillum dolore eu fugiat nulla pariatur. Excepteur sint occaecat cupidatat non proident, sunt in culpa qui officia deserunt mollit anim id est laborum.

% --------------------------------------------------------------------------------
% ----- Todo list
% --------------------------------------------------------------------------------
\listoftodos
% \todototoc

\end{document}
